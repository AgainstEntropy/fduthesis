\documentclass{fdudoc}
\usepackage{multirow,xpinyin}
\hypersetup{
  pdftitle  = {fduthesis: 复旦大学论文模板},
  pdfauthor = {曾祥东}}
% 全角标点放在引号中,需要改成半角式,否则间距过大,不好看
\def\FSID{“{\xeCJKsetup{PunctStyle=banjiao}。}”} % U+3002
\def\FSFW{“{\xeCJKsetup{PunctStyle=banjiao}.}”} % U+FF0E
\def\COFW{“{\xeCJKsetup{PunctStyle=banjiao}:}”} % U+FF1A
\def\SCFW{“{\xeCJKsetup{PunctStyle=banjiao};}”} % U+FF1B

\title{\textcolor{MaterialIndigo800}{%
  \textbf{fduthesis: 复旦大学论文\xpinyin[font=\sffamily]{模}{mu2}板}}}
\author{曾祥东}
\date{2019/04/03\quad v0.7d%
  \thanks{\url{https://github.com/stone-zeng/fduthesis}.}}

\begin{document}

% 禁止使用 " 符号作为抄录文本缩略符
\DeleteShortVerb\"

% 封面与目录的页边距
\newgeometry{
  left   = 1.25 in,
  right  = 1.25 in,
  top    = 1.25 in,
  bottom = 1.00 in
}

\maketitle
\vfill
\begin{center}
  \includegraphics[width=8cm]{../logo/fduthesis-cover.pdf}
\end{center}
\vfill
\thispagestyle{plain}
\clearpage

\tableofcontents

% 用户手册的页边距
\newgeometry{
  left   = 1.65 in,
  right  = 0.80 in,
  top    = 1.25 in,
  bottom = 1.00 in
}

\section{介绍}

目前,在网上可以找到的复旦大学 \LaTeX{} 论文模板主要有以下这些:
\begin{itemize}
  \item 数学科学学院 2001 级的何力同学和李湛同学在 2005 年根据
    学校要求所设计的 \cls{毕业论文格式 tex04 版},以及 2008 年
    张越同学修改之后的 \cls{毕业论文格式 tex08 版},这是专为
    数院本科生撰写毕业论文而设计的
    \scite{数院毕业论文格式,数院毕业论文格式更新};
  \item Pandoxie 编写的 \cls{FDU-Thesis-Latex}
    \scite{pandoxie2014fduthesislatex},基本满足了博士(硕士)
    毕业论文格式要求,使用人数较多;
  \item richarddzh 编写的硕士论文模板 \cls{fudan-thesis}
    \scite{richard2016fudanthesis}。
\end{itemize}
以上这些模板大都没有经过系统的设计,也鲜有后续维护。相比之下,
清华大学 \scite{thuthesis}、重庆大学 \scite{cquthesis}、
中国科学技术大学 \scite{ustcthesis} 中国科学院大学 \scite{ucasthesis}
以及友校上海交通大学 \scite{sjtuthesis}等,都有成熟、
稳定的解决方案,值得参考。

本模板将借鉴前辈经验,重新设计,并使用 \LaTeX3
\scite{source3} 编写,以适应 \TeX{} 技术发展潮流;
同时还将构建一套简洁的接口,方便用户使用。

\subsection*{\LaTeX{} 入门}

本文档并非是一份 \LaTeX{} 零基础教程。如果您是完完全全的新手,
建议先阅读相关入门文档,如刘海洋编著的《\LaTeX{} 入门》
\scite{刘海洋2013latex入门} 第一章,或大名鼎鼎的“\pkg{lshort}”
\scite{lshort} 及其中文翻译版 \scite{lshort-zh-cn}。当然,
网络上的入门教程多如牛毛,您可以自行选取。

\subsection*{关于本文档}

本文采用不同字体表示不同内容。无衬线字体表示宏包名称,如
\pkg{xeCJK} 宏包、\cls{fduthesis} 文档类等;等宽字体表示代码或
文件名,如 \cs{fdusetup} 命令、\env{abstract} 环境、\TeX{} 文档
\file{thesis.tex} 等;带有尖括号的楷体(或西文斜体)表示命令参数,
如 \meta{模板选项}、\meta{English title} 等。在使用时,参数两侧
的尖括号不必输入。示例代码进行了语法高亮处理,以方便阅读。

在用户手册中,带有蓝色侧边线的为 \LaTeX{} 代码,而带有粉色侧边线
的则为命令行代码,请注意区分。模板提供的选项、命令、环境等,
均用横线框起,同时给出使用语法和相关说明。

本模板中的选项、命令或环境可以分为以下三类:
\begin{itemize}
  \item 名字后面带有 \rexptarget\rexpstar{} 的,表示只能在^^A
    \emph{中文模板}中使用;
  \item 名字后面带有 \exptarget\expstar{} 的,表示只能在^^A
    \emph{英文模板}中使用;
  \item 名字后面不带有特殊符号的,表示既可以在中文模板中使用,
    也可以在英文模板中使用。
\end{itemize}

代码实现主要面向对 \LaTeX{} 宏包开发感兴趣的用户。如果您有任何改进
意见或者功能需求,欢迎前往 GitHub 仓库
\href{https://github.com/stone-zeng/fduthesis/issues}{提交 issue}。

文档的最后还提供了版本历史和代码索引,以供查阅。

\section{安装}

\subsection{获取 \cls{fduthesis}}

\subsubsection{标准安装}

如果没有特殊理由,始终建议您使用宏包管理器安装 \cls{fduthesis}。
例如在 \TeXLive{} 中,执行(可能需要管理员权限)
\begin{shellexample}[morekeywords={tlmgr,install}]
  tlmgr install fduthesis
\end{shellexample}
即可完成安装。

在 \TeXLive{} 和 \MiKTeX{} 中,您还可以通过图形界面进行安装,
此处不再赘述。

\subsubsection{手动安装}

如果您需要从 CTAN 上自行下载并手动安装,较好的方法是使用 TDS
安装包:
\begin{itemize}
  \item 从 CTAN 上下载 \cls{fduthesis} 的
    \href{http://mirror.ctan.org/install/macros/latex/contrib/fduthesis.tds.zip}{TDS 安装包};
  \item 按目录结构将 \file{fduthesis.tds.zip} 中的文件复制到 \TeX{}
    发行版的本地 TDS 根目录;
  \item 执行 \bashcmd{mktexlsr} 刷新文件名数据库以完成安装。
\end{itemize}
%
您也可以从源代码直接生成模板(不推荐):
\begin{itemize}
  \item 打开 \href{https://github.com/stone-zeng/fduthesis}^^A
    {项目主页},点击“Clone or download”,并选择“Download ZIP”,
    下载 \file{fduthesis-master.zip};如果您的电脑中安装有 git
    程序,也可通过以下命令直接克隆代码仓库:
    \begin{shellexample}[gobble=7,alsoletter={.},^^A
        morekeywords={git,clone}]
      git clone https://github.com/stone-zeng/fduthesis.git
    \end{shellexample}
  \item 解压并进入到 \file{source} 文件夹,执行以下命令以生成
    模板的各组件:
    \begin{shellexample}[gobble=7,morekeywords={xetex}]
      xetex fduthesis.dtx
    \end{shellexample}
  \item 将生成的文档类(\file{.cls})、宏包(\file{.sty})以及
    参数配置文件(\file{.def})复制到 \TeX{} 发行版本地 TDS 树
    的 \path{texmf-local/tex/latex/fduthesis/} 目录下,并执行
    \bashcmd{mktexlsr} 刷新文件名数据库,方可完成安装。
  \item 使用 \cls{fduthesis} 撰写论文时,您还需要从代码仓库下的
    \file{testfiles/support} 目录中复制 \file{fudan-name.pdf}
    文件至工作目录,以确保封面中的校名图片可以正确显示。
\end{itemize}

\subsubsection{扁平化安装}

如果您不希望安装本模板,但需要立刻使用,也可以使用模板提供的安装脚本。
从 GitHub 上获取代码仓库后,执行 \file{install-win.bat}(Windows 系统)
或 \file{install-linux.sh}(Linux 系统),所有需要的文件便会在
\file{thesis} 文件夹中生成。

\subsection{模板组成}

本模板主要包含核心文档类、配置文件、附属宏包以及用户文档等几个
部分,其具体组成见表~\ref{tab:fduthesis-components}。

\begin{table}[ht]
  \caption{\cls{fduthesis} 的主要组成部分}
  \label{tab:fduthesis-components}
  \centering
  \begin{tabular}{lp{20em}}
    \toprule
    \textbf{文件} & \textbf{功能说明} \\
    \midrule
    \file{fduthesis.cls}          & 中文模板文档类 \\
    \file{fduthesis-en.cls}       & 英文模板文档类 \\
    \file{fduthesis.def}          & 参数配置文件,用于设定
      \cls{fduthesis} 的初始参数,不建议您自行改动 \\
    \file{fdudoc.cls}             & 用户手册文档类 \\
    \file{fdulogo.sty}            & 复旦大学视觉识别系统 \\
    \file{fudan-emblem.pdf}       & 校徽 \\
    \file{fudan-emblem-new.pdf}   & 校徽(重修版) \\
    \file{fudan-name.pdf}         & 校名图片 \\
    \file{README.md}              & 简要自述 \\
    \ifdefined\FDUCODEDOC
      \file{fduthesis.pdf}        & 中文用户手册 \\
      \file{fduthesis-en.pdf}     & 英文用户手册 \\
      \file{fduthesis-code.pdf}   & 模板实现代码(本文档) \\
    \else
      \file{fduthesis.pdf}        & 中文用户手册(本文档) \\
      \file{fduthesis-en.pdf}     & 英文用户手册 \\
      \file{fduthesis-code.pdf}   & 模板实现代码 \\
    \fi
    \file{fduthesis-template.tex} & 空白模板,可据此为基础完成论文
      撰写 \\
    \bottomrule
  \end{tabular}
\end{table}

\section{使用说明}

\subsection{基本用法}

以下是一份简单的 \TeX{} 文档,它演示了 \cls{fduthesis}
的最基本用法:
\begin{latexexample}[deletetexcs={\documentclass},
    moretexcs={\chapter},morekeywords={\documentclass},
    emph={[2]document}]
  thesis.tex
  \documentclass{fduthesis}
  \begin{document}
    \chapter{欢迎}
    \section{Welcome to fduthesis!}
    你好,\LaTeX{}!
  \end{document}
\end{latexexample}

按照 \ref{subsec:编译方式}~小节中的方式编译该文档,您应当得到
一篇 5 页的文章。当然,这篇文章的绝大部分都是空白的。

英文模板可以用类似的方式使用:
\begin{latexexample}[deletetexcs={\documentclass},
    moretexcs={\chapter},morekeywords={\documentclass},
    emph={[2]document}]
  thesis-en.tex
  \documentclass{fduthesis-en}
  \begin{document}
    \chapter{Welcome}
    \section{Welcome to fduthesis!}
    Hello, \LaTeX{}!
  \end{document}
\end{latexexample}
英文模板只对正文部分进行了改动,封面、指导小组成员以及声明页仍将
显示为中文。

\subsection{编译方式} \label{subsec:编译方式}

本模板不支持 \pdfTeX{} 引擎,请使用 \XeLaTeX{} 或 \LuaLaTeX{}
编译。推荐使用 \XeLaTeX{}。为了生成正确的目录、脚注以及交叉引用,
您至少需要连续编译两次。

以下代码中,假设您的 \TeX{} 源文件名为 \file{thesis.tex}。
使用 \XeLaTeX{} 编译论文,请在命令行中执行
\begin{shellexample}[morekeywords={xelatex}]
  xelatex thesis
  xelatex thesis
\end{shellexample}
或使用 \pkg{latexmk}:
\begin{shellexample}[morekeywords={latexmk},emph={-xelatex}]
  latexmk -xelatex thesis
\end{shellexample}

使用 \LuaLaTeX{} 编译论文,请在命令行中执行
\begin{shellexample}[morekeywords={lualatex}]
  lualatex thesis
  lualatex thesis
\end{shellexample}
或者
\begin{shellexample}[morekeywords={latexmk},emph={-lualatex}]
  latexmk -lualatex thesis
\end{shellexample}

\subsection{模板选项}

所谓“模板选项”,指需要在引入文档类的时候指定的选项:
\begin{latexexample}[deletetexcs={\documentclass},
    morekeywords={\documentclass}]
  \documentclass(*\oarg{模板选项}*){fduthesis}
  \documentclass(*\oarg{模板选项}*){fduthesis-en}
\end{latexexample}

有些模板选项为布尔型,它们只能在 \opt{true} 和 \opt{false}
中取值。对于这些选项,\kvopt{\meta{选项}}{true} 中的“|= true|”
可以省略。

\begin{function}[added=2018-02-01]{type}
  \begin{fdusyntax}[emph={[1]type}]
    type = (*<doctor|master|(bachelor)>*)
  \end{fdusyntax}
  选择论文类型。三种选项分别代表博士学位论文、硕士学位论文和本科
  毕业论文。
\end{function}

\begin{function}{oneside,twoside}
  指明论文的单双面模式,默认为 \opt{twoside}。该选项会影响每章
  的开始位置,还会影响页眉样式。
\end{function}

在双面模式(\opt{twoside})下,按照通常的排版惯例,每章应只从
奇数页(在右)开始;而在单页模式(\opt{oneside})下,则可以从
任意页面开始。本模板中,目录、摘要、符号表等均视作章,也按相同
方式排版。

双面模式下,正文部分偶数页(在左)的左页眉显示章标题,奇数页
(在右)的右页眉显示节标题;前置部分的页眉按同样格式显示,但文字
均为对应标题(如“目录”、“摘要”等)。
而在单面模式下,正文部分则页面不分奇偶,均同时显示左、右页眉,
文字分别为章标题和节标题;前置部分只有中间页眉,显示对应标题。

\begin{function}{draft}
  \begin{fdusyntax}[emph={[1]draft}]
    draft = (*<\TFF>*)
  \end{fdusyntax}
  选择是否开启草稿模式,默认关闭。
\end{function}

草稿模式为全局选项,会影响到很多宏包的工作方式。
开启之后,主要的变化有:
\begin{itemize}
  \item 把行溢出的盒子显示为黑色方块;
  \item 不实际插入图片,只输出一个占位方框;
  \item 关闭超链接渲染,也不再生成 PDF 书签;
  \item 显示页面边框。
\end{itemize}

\begin{function}[added=2018-01-31]{config}
  \begin{fdusyntax}[emph={[1]config}]
    config = (*\marg{文件}*)
  \end{fdusyntax}
  用户配置文件的文件名。默认为空,即不载入用户配置文件。
\end{function}

\subsection{参数设置}

\begin{function}{\fdusetup}
  \begin{fdusyntax}[morekeywords={\fdusetup}]
    \fdusetup(*\marg{键值列表}*)
  \end{fdusyntax}
  本模板提供了一系列选项,可由您自行配置。载入文档类之后,以下
  所有选项均可通过统一的命令 \cs{fdusetup} 来设置。
\end{function}

\cs{fdusetup} 的参数是一组由(英文)逗号隔开的选项列表,列表中的
选项通常是 \kvopt{\meta{key}}{\meta{value}} 的形式。部分选项的
\meta{value} 可以省略。对于同一项,后面的设置将会覆盖前面的设置。
在下文的说明中,将用\textbf{粗体}表示默认值。

\cs{fdusetup} 采用 \LaTeX3 风格的键值设置,支持不同类型以及多种
层次的选项设定。键值列表中,“|=|”左右的空格不影响设置;但需注意,
参数列表中不可以出现空行。

与模板选项相同,布尔型的参数可以省略 \kvopt{\meta{选项}}{true}
中的“|= true|”。

另有一些选项包含子选项,如 \opt{style} 和 \opt{info} 等。它们可以
按如下两种等价方式来设定:
\begin{latexexample}[morekeywords={\fdusetup},
    emph={[1]style,cjk-font,font-size,info,title,title*,author,author*,department}]
  \fdusetup{
    style = {cjk-font = adobe, font-size = -4},
    info  = {
      title      = {论动体的电动力学},
      title*     = {On the Electrodynamics of Moving Bodies},
      author     = {阿尔伯特·爱因斯坦},
      author*    = {Albert Einstein},
      department = {物理学系}
    }
  }
\end{latexexample}
或者
\begin{latexexample}[morekeywords={\fdusetup},
    emph={[1]style,cjk-font,font-size,info,title,title*,author,author*,department}]
  \fdusetup{
    style/cjk-font  = adobe,
    style/font-size = -4,
    info/title      = {论动体的电动力学},
    info/title*     = {On the Electrodynamics of Moving Bodies},
    info/author     = {阿尔伯特·爱因斯坦},
    info/author*    = {Albert Einstein},
    info/department = {物理学系}
  }
\end{latexexample}

注意 “|/|” 的前后均不可以出现空白字符。

\subsubsection{论文格式} \label{subsubsec:论文格式}

\begin{function}{style}
  \begin{fdusyntax}[emph={[1]style}]
    style = (*\marg{键值列表}*)
    style/(*\meta{key}*) = (*\meta{value}*)
  \end{fdusyntax}
  该选项包含许多子项目,用于设置论文格式。具体内容见下。
\end{function}

\begin{function}[updated=2019-03-05]{style/font}
  \begin{fdusyntax}[emph={[1]font}]
    font = (*<garamond|libertinus|lm|palatino|(times)|times*|none>*)
  \end{fdusyntax}
  设置西文字体(包括数学字体)。具体配置见表~\ref{tab:font}。
\end{function}

\begin{table}[ht]
\begin{threeparttable}
  \caption{西文字体配置}
  \label{tab:font}
  \centering
  \begin{tabular}{ccccc}
    \toprule
      & \strong{正文字体} & \strong{无衬线字体} & \strong{等宽字体} & \strong{数学字体} \\
    \midrule
      |garamond|        & EB Garamond         & Libertinus Sans & LM Mono\tnote{a} & Garamond Math   \\
      |libertinus|      & Libertinus Serif    & Libertinus Sans & LM Mono          & Libertinus Math \\
      |lm|              & LM Roman            & LM Sans         & LM Mono          & LM Math         \\
      |palatino|        & TG Pagella\tnote{b} & Libertinus Sans & LM Mono          & TG Pagella Math \\
      |times|           & XITS                & TG Heros        & TG Cursor        & XITS Math       \\
      |times*|\tnote{c} & Times New Roman     & Arial           & Courier New      & XITS Math       \\
    \bottomrule
  \end{tabular}
  \begin{tablenotes}
    \item[a] “LM”是 Latin Modern 的缩写。
    \item[b] “TG”是 TeX Gyre 的缩写。
    \item[c] 本行中,Times New Roman、Arial 和 Courier New 是商业字体,
      在 Windows 和 macOS 系统上均默认安装。
  \end{tablenotes}
\end{threeparttable}
\end{table}

\begin{function}[rEXP,updated=2019-03-05]{style/cjk-font}
  \begin{fdusyntax}[emph={[1]cjk-font}]
    cjk-font = (*<adobe|(fandol)|founder|mac|sinotype|sourcehan|windows|none>*)
  \end{fdusyntax}
  设置中文字体。具体配置见表~\ref{tab:cjk-font}。
\end{function}

\begin{table}[ht]
  \caption{中文字体配置}
  \label{tab:cjk-font}
  \centering
  \begin{tabular}{cccc}
    \toprule
      & \strong{正文字体(宋体)} & \strong{无衬线字体(黑体)} & \strong{等宽字体(仿宋)} \\
    \midrule
      \multirow{2}*{|adobe|}     & Adobe 宋体          & Adobe  黑体        & Adobe  仿宋        \\
                                 & Adobe Song Std      & Adobe Heiti Std    & Adobe Fangsong Std \\
      \multirow{2}*{|fandol|}    & Fandol 宋体         & Fandol 黑体        & Fandol 仿宋        \\
                                 & FandolSong          & FandolHei          & FandolFang         \\
      \multirow{2}*{|founder|}   & 方正书宋            & 方正黑体           & 方正仿宋           \\
                                 & FZShuSong-Z01       & FZHei-B01          & FZFangSong-Z02     \\
      \multirow{2}*{|mac|}       & (华文)宋体-简     & (华文)黑体-简    & 华文仿宋           \\
                                 & Songti SC           & Heiti SC           & STFangsong         \\
      \multirow{2}*{|sinotype|}  & 华文宋体            & 华文黑体           & 华文仿宋           \\
                                 & STSong              & STHeiti            & STFangsong         \\
      \multirow{2}*{|sourcehan|} & 思源宋体            & 思源黑体           & ---                \\
                                 & Source Han Serif SC & Source Han Sans SC & ---                \\
      \multirow{2}*{|windows|}   & (中易)宋体        & (中易)黑体       & (中易)仿宋       \\
                                 & SimSun              & SimHei             & FangSong           \\
    \bottomrule
  \end{tabular}
\end{table}

启用 \kvopt{font}{none} 或 \kvopt{cjk-font}{none} 之后,模板将关闭
默认西文 / 中文字体设置。此时,您需要自行使用 \cs{setmainfont}、
\cs{setCJKmainfont}、\cs{setmathfont} 等命令来配置字体。

\begin{function}{style/font-size}
  \begin{fdusyntax}[emph={[1]font-size}]
    font-size = (*<(-4)|5>*)
  \end{fdusyntax}
  设置论文的基础字号。
\end{function}

\begin{function}[rEXP,updated=2017-10-14]{style/fullwidth-stop}
  \begin{fdusyntax}[emph={[1]fullwidth-stop}]
    fullwidth-stop = (*<catcode|mapping|(false)>*)
  \end{fdusyntax}
  选择是否把全角实心句点\FSFW 作为默认的句号形状。
  这种句号一般用于科技类文章,以避免与下标“$_o$”或“$_0$”混淆。
\end{function}

选择 \kvopt{fullwidth-stop}{catcode} 或 \opt{mapping} 后,都会实现
上述效果。有所不同的是,在选择 \opt{catcode} 后,只有^^A
\emph{显式的}\FSID 会被替换为\FSFW;但在选择 \opt{mapping} 后,
\emph{所有的}\FSID 都会被替换。例如,如果您用宏保存了一些含有^^A
\FSID 的文字,那么在选择 \opt{catcode} 时,其中的\FSID 不会被
替换为\FSFW。

选项 \kvopt{fullwidth-stop}{mapping} 只在 \XeTeX{} 下有效。使用
\LuaTeX{} 编译时,该选项相当于 \kvopt{fullwidth-stop}{catcode}。

如果您在选择 \kvopt{fullwidth-stop}{mapping} 后仍需要临时显示^^A
\FSID,可以按如下方法操作:
\begin{latexexample}[moretexcs={\CJKfontspec},emph={[1]Mapping}]
  请使用 XeTeX 编译
  外侧的花括号表示分组
  这是一个句号{\CJKfontspec{(*\meta{字体名}*)}[Mapping=full-stop]。}
\end{latexexample}

\begin{function}{style/footnote-style}
% 这里奇怪的东西是用来控制对齐的。fdusyntax 会吃掉开头的几个
% 空格,因此这里用 X 来占位。
  \begin{fdusyntax}[emph={[1]footnote-style}]
    footnote-style = (*<plain|\\
      XXXXXX\mbox{}~~~~~~~~~~~~~~~~~libertinus|libertinus*|libertinus-sans|\\
      XXXXXX\mbox{}~~~~~~~~~~~~~~~~~pifont|pifont*|pifont-sans|pifont-sans*|\\
      XXXXXX\mbox{}~~~~~~~~~~~~~~~~~xits|xits-sans|xits-sans*>*)
  \end{fdusyntax}
  设置脚注编号样式。西文字体设置会影响其默认取值(见
  表~\ref{tab:footnote-font})。因此,要使得该选项生效,需将其
  放置在 \opt{font} 选项之后。带有 |sans| 的为相应的无衬线字体
  版本;带有 |*| 的为阴文样式(即黑底白字)。
\end{function}

\begin{table}[ht]
  \caption{西文字体与脚注编号样式默认值的对应关系}
  \label{tab:footnote-font}
  \centering
  \begin{tabular}{ccccc}
    \toprule
    \textbf{西文字体设置} &
      |libertinus| & |lm|     & |palatino| & |times| \\
    \midrule
    \textbf{脚注编号样式默认值} &
      |libertinus| & |pifont| & |pifont|   & |xits|  \\
    \bottomrule
  \end{tabular}
\end{table}

\begin{function}[added=2017-08-13]{style/hyperlink}
  \begin{fdusyntax}[emph={[1]hyperlink}]
    hyperlink = (*<border|(color)|none>*)
  \end{fdusyntax}
  设置超链接样式。\opt{border} 表示在超链接四周绘制方框;
  \opt{color} 表示用彩色显示超链接;\opt{none} 表示没有特殊装饰,
  可用于生成最终的打印版文稿。
\end{function}

\begin{function}[added=2017-08-13,updated=2017-12-08]{style/hyperlink-color}
  \begin{fdusyntax}[emph={[1]hyperlink-color}]
    hyperlink-color = (*<(default)|classic|elegant|fantasy|material|\\
      XXXXXX\mbox{}~~~~~~~~~~~~~~~~~~business|science|summer|autumn|graylevel|prl>*)
  \end{fdusyntax}
  设置超链接颜色。该选项在 \kvopt{hyperlink}{none} 时无效。
  各选项所代表的颜色见表~\ref{tab:hyperlink-color}。
\end{function}

\begin{table}[ht]
\centering
\newcommand\linkcolorexam[3]{^^A
  {\small 图~\textcolor[HTML]{#1}{1-2},
    (\textcolor[HTML]{#1}{3.4})~式} &
  {\small \textcolor[HTML]{#2}{\texttt{http://g.cn}}} &
  {\small 文献~[\textcolor[HTML]{#3}{1}],
    (\textcolor[HTML]{#3}{Knuth~1986})}}
\begin{threeparttable}
\caption{预定义的超链接颜色方案}
\label{tab:hyperlink-color}
\begin{tabular}{c*{3}{>{\hspace{0.2cm}}c<{\hspace{0.2cm}}}}
  \toprule
  \textsf{选项} & \textsf{链接} & \textsf{URL} & \textsf{引用} \\
  \midrule
  \opt{default}            & \linkcolorexam{990000}{0000B2}{007F00} \\
  \opt{classic}            & \linkcolorexam{FF0000}{0000FF}{00FF00} \\
  \opt{elegant}\tnote{a}   & \linkcolorexam{961212}{C31818}{9B764F} \\
  \opt{fantasy}\tnote{b}   & \linkcolorexam{FF4A19}{FF3F94}{934BA1} \\
  \opt{material}\tnote{c}  & \linkcolorexam{E91E63}{009688}{4CAF50} \\
  \opt{business}\tnote{d}  & \linkcolorexam{D14542}{295497}{1F6E43} \\
  \opt{science}\tnote{e}   & \linkcolorexam{CA0619}{389F9D}{FF8920} \\
  \opt{summer}\tnote{f}    & \linkcolorexam{00AFAF}{5F5FAF}{5F8700} \\
  \opt{autumn}\tnote{f}    & \linkcolorexam{D70000}{D75F00}{AF8700} \\
  \opt{graylevel}\tnote{c} & \linkcolorexam{616161}{616161}{616161} \\
  \opt{prl}\tnote{g}       & \linkcolorexam{2D3092}{2D3092}{2D3092} \\
  \bottomrule
\end{tabular}
\begin{tablenotes}
  \item[a] 来自 \href{https://tex.stackexchange.com/}^^A
    {\TeX{} - \LaTeX{} Stack Exchange 网站}。
  \item[b] Adobe CC 产品配色。
  \item[c] 取自 Material 色彩方案
    (见 \url{https://material.io/guidelines/style/color.html})。
  \item[d] Microsoft Office 2016 产品配色。
  \item[e] 来自 \href{https://www.wolfram.com/}{Wolfram Research 网站}。
  \item[f] 均取自 Solarized 色彩方案
    (见 \url{http://ethanschoonover.com/solarized})。
  \item[g] \textit{Physical Review Letter} 杂志配色。
\end{tablenotes}
\end{threeparttable}
\end{table}

\begin{function}[added=2018-01-25]{style/bib-backend}
  \begin{fdusyntax}[emph={[1]bib-backend}]
    bib-backend = (*<bibtex|biblatex>*)
  \end{fdusyntax}
  选择参考文献的支持方式。选择 \opt{bibtex} 后,将使用 \BibTeX{}
  处理文献,样式由 \pkg{natbib} 宏包负责;选择 \opt{biblatex} 后,
  将使用 \biber{} 处理文献,样式则由 \pkg{biblatex} 宏包负责。
\end{function}

\begin{function}[added=2017-10-28,updated=2018-01-25]^^A
    {style/bib-style}
  \begin{fdusyntax}[emph={[1]bib-style}]
    bib-style = (*<author-year|(numerical)|\meta{其他样式}>*)
  \end{fdusyntax}
  设置参考文献样式。\opt{author-year} 和 \opt{numerical} 分别对应
  国家标准 GB/T 7714--2015 \scite{gb-t-7714-2015} 中的著者—出版年制
  和顺序编码制。选择 \meta{其他样式} 时,如果 \kvopt{bib-backend}^^A
  {bibtex},需保证相应的 \file{.bst} 格式文件能被调用;而如果
  \kvopt{bib-backend}{biblatex},则需保证相应的 \file{.bbx} 格式文件
  能被调用。
\end{function}

\begin{function}[added=2018-01-25]{style/cite-style}
  \begin{fdusyntax}[emph={[1]cite-style}]
    cite-style = (*\marg{引用样式}*)
  \end{fdusyntax}
  选择引用格式。默认为空,即与参考文献样式(著者—出版年制或顺序
  编码制)保持一致。如果手动填写,需保证相应的 \file{.cbx} 格式文件
  能被调用。该选项在 \kvopt{bib-backend}{bibtex} 时无效。
\end{function}

\begin{function}[added=2018-01-25]{style/bib-resource}
  \begin{fdusyntax}[emph={[1]bib-resource}]
    bib-resource = (*\marg{文件}*)
  \end{fdusyntax}
  参考文献数据源。可以是单个文件,也可以是用英文逗号隔开的一组文件。
  如果 \kvopt{bib-backend}{biblatex},则必须明确给出 \file{.bib}
  后缀名。
\end{function}

\begin{function}[added=2017-08-10]{style/logo}
  \begin{fdusyntax}[emph={[1]logo}]
    logo = (*\marg{文件}*)
  \end{fdusyntax}
  封面中校名图片的文件名。默认值为 \file{fudan-name.pdf}。
\end{function}

\begin{function}[added=2017-08-10]{style/logo-size}
  \begin{fdusyntax}[emph={[1]logo-size}]
    logo-size = (*\marg{宽度}*)
    logo-size = {(*\meta{宽度}*), (*\meta{高度}*)}
  \end{fdusyntax}
  校名图片的大小。默认仅指定了宽度,为 |0.5\textwidth|\/。
  如果仅需指定高度,可在 \meta{宽度} 处填入一个空的分组 |{}|。
\end{function}

\begin{function}[added=2017-07-06]{style/auto-make-cover}
  \begin{fdusyntax}[emph={[1]auto-make-cover}]
    auto-make-cover = (*<\TTF>*)
  \end{fdusyntax}
  是否自动生成论文封面(封一)、指导小组成员名单(封二)和
  声明页(封三)。封面中的各项信息,可通过 \cs{fdusetup} 录入,
  具体请参阅 \ref{subsubsec:信息录入}~节。
\end{function}

\begin{function}{\makecoveri,\makecoverii,\makecoveriii}
  用于手动生成论文封面、指导小组成员名单和声明页。这几个命令不能
  确保页码的正确编排,因此除非必要,您应当始终使用自动生成的封面。
\end{function}

\subsubsection{信息录入} \label{subsubsec:信息录入}

\begin{function}{info}
  \begin{fdusyntax}[emph={[1]info}]
    info = (*\marg{键值列表}*)
    info/(*\meta{key}*) = (*\meta{value}*)
  \end{fdusyntax}
  该选项包含许多子项目,用于录入论文信息。具体内容见下。以下带“|*|”
  的项目表示对应的英文字段。
\end{function}

\begin{function}[added=2018-02-01,updated=2019-03-12]{info/degree}
  \begin{fdusyntax}[emph={[1]degree}]
    degree = (*<(academic)|professional>*)
  \end{fdusyntax}
  学位类型,仅适用于博士和硕士学位论文。\opt{academic} 和 \opt{professional}
  分别表示学术学位和专业学位。
\end{function}

\begin{function}{info/title,info/title*}
  \begin{fdusyntax}[emph={[1]title,title*}]
    title  = (*\marg{中文标题}*)
    title* = (*\marg{英文标题}*)
  \end{fdusyntax}
  论文标题。默认会在约 20 个汉字字宽处强制断行,但为了语义的
  连贯以及排版的美观,如果您的标题长于一行,建议使用“|\\|”
  手动断行。
\end{function}

\begin{function}{info/author,info/author*}
  \begin{fdusyntax}[emph={[1]author,author*}]
    author  = (*\marg{姓名}*)
    author* = (*\marg{英文姓名(或拼音)}*)
  \end{fdusyntax}
  作者姓名。
\end{function}

\begin{function}{info/supervisor}
  \begin{fdusyntax}[emph={[1]supervisor}]
    supervisor = (*\marg{姓名}*)
  \end{fdusyntax}
  导师姓名。
\end{function}

\begin{function}{info/department}
  \begin{fdusyntax}[emph={[1]department}]
    department = (*\marg{名称}*)
  \end{fdusyntax}
  院系名称。
\end{function}

\begin{function}{info/major}
  \begin{fdusyntax}[emph={[1]major}]
    major = (*\marg{名称}*)
  \end{fdusyntax}
  专业名称。
\end{function}

\begin{function}{info/student-id}
  \begin{fdusyntax}[emph={[1]student-id}]
    student-id = (*\marg{数字}*)
  \end{fdusyntax}
  作者学号。
\end{function}

复旦大学学号共 11 位,前两位为入学年份,之后一位为学生类型
代码(博士生为 1,硕士生为 2,本科生为 3),接下来的五位为
专业代码,最后三位为顺序号。

\begin{function}{info/school-id}
  \begin{fdusyntax}[emph={[1]school-id}]
    school-id = (*\marg{数字}*)
  \end{fdusyntax}
  学校代码。默认值为 10246(这是复旦大学的学校代码)。
\end{function}

\begin{function}{info/date}
  \begin{fdusyntax}[emph={[1]date}]
    date = (*\marg{日期}*)
  \end{fdusyntax}
  论文完成日期。默认值为文档编译日期(\tn{today})。
\end{function}

\begin{function}[added=2017-07-04]{info/secret-level}
  \begin{fdusyntax}[emph={[1]secret-level}]
    secret-level = (*<(none)|i|ii|iii>*)
  \end{fdusyntax}
  密级。\opt{i}、\opt{ii}、\opt{iii} 分别表示秘密、机密、绝密;
  \opt{none} 表示论文不涉密,即不显示密级与保密年限。
\end{function}

\begin{function}[added=2017-07-04]{info/secret-year}
  \begin{fdusyntax}[emph={[1]secret-year}]
    secret-year = (*\marg{年限}*)
  \end{fdusyntax}
  保密年限。建议您使用中文,如“五年”。该选项在设置
  \kvopt{secret-level}{none} 时无效。
\end{function}

\begin{function}{info/instructors}
  \begin{fdusyntax}[emph={[1]instructors}]
    instructors = (*\marg{成员 1, 成员 2, ...}*)
  \end{fdusyntax}
  指导小组成员。各成员之间需使用英文逗号隔开。为防止歧义,
  可以用分组括号“|{...}|”把各成员字段括起来。
\end{function}

\begin{function}{info/keywords,info/keywords*}
  \begin{fdusyntax}[emph={[1]keywords,keywords*}]
    keywords  = (*\marg{中文关键字}*)
    keywords* = (*\marg{英文关键字}*)
  \end{fdusyntax}
  关键字列表。各关键字之间需使用英文逗号隔开。为防止歧义,
  可以用分组括号“|{...}|”把各字段括起来。
\end{function}

\begin{function}{info/clc}
  \begin{fdusyntax}[emph={[1]clc}]
    clc = (*\marg{分类号}*)
  \end{fdusyntax}
  中图分类号(CLC)。
\end{function}

\subsection{正文编写}

\begin{quotation}
  喬孟符(吉)博學多能,以樂府稱。嘗云:「作樂府亦有法,曰^^A
  \CJKunderdot{鳳頭、豬肚、豹尾}六字是也。」大概起要美麗,中要浩蕩,
  結要響亮。尤貴在首尾貫穿,意思清新。苟能若是,斯可以言樂府矣。
\end{quotation}
\hfill ——陶宗儀《南村輟耕錄·作今樂府法》

\subsubsection{凤头}

\begin{function}{\frontmatter}
  声明前置部分开始。
\end{function}

在本模板中,前置部分包含目录、中英文摘要以及符号表等。
前置部分的页码采用小写罗马字母,并且与正文分开计数。

\begin{function}{\tableofcontents,\listoffigures,\listoftables}
  生成目录。为了生成完整、正确的目录,您至少需要编译\emph{两次}。对于图表
  较多的论文,也可以使用 \cs{listoffigures} 和 \cs{listoftables} 生成单独的
  插图、表格目录。
\end{function}

% TODO: \DescribeEnv{abstract}
% TODO: \DescribeEnv{abstract*}
\begin{function}{abstract}
  \begin{fdusyntax}[emph={[2]abstract}]
    中文论文模板 (fduthesis)      英文论文模板 (fduthesis-en)
    \begin{abstract}                \begin{abstract}
      (*\meta{中文摘要} \hspace{3.52cm} \meta{英文摘要}*)
    \end{abstract}                  \end{abstract}
  \end{fdusyntax}
\end{function}
\begin{function}[rEXP]{abstract*}
  \begin{fdusyntax}[emph={[2]abstract*}]
    中文论文模板 (fduthesis)
    \begin{abstract*}
      (*\meta{英文摘要}*)
    \end{abstract*}
  \end{fdusyntax}
  摘要。中文模板中,不带星号和带星号的版本分别用来输入中文摘要
  和英文摘要;英文模板中没有带星号的版本,您只需输入英文摘要。
\end{function}

摘要的最后,会显示关键字列表以及中图分类号(CLC)。
这两项可通过 \cs{fdusetup} 录入,具体
请参阅 \ref{subsubsec:信息录入}~节。

% TODO: \DescribeEnv{notation}
\begin{function}{notation}
  \begin{fdusyntax}[emph={[2]notation}]
    \begin{notation}(*\oarg{列格式说明}*)
      (*\meta{符号 1}*)  &  (*\meta{说明}*)  \\
      (*\meta{符号 2}*)  &  (*\meta{说明}*)  \\
      (*\phantom{\meta{符号 $n$}}*)  (*$\vdots$*)
      (*\meta{符号\ \kern-0.1em$n$}*)  &  (*\meta{说明}*)
    \end{notation}
  \end{fdusyntax}
  符号表。可选参数 \meta{列格式说明}与 \LaTeX{} 中标准表格的列格
  式说明语法一致,默认值为“|lp{7.5cm}|”,即第一列宽度自动调整,
  第二列限宽 \SI{7.5}{cm},两列均为左对齐。
\end{function}

\subsubsection{猪肚}

\begin{function}{\mainmatter}
  声明主体部分开始。
\end{function}

主体部分是论文的核心,您可以分章节撰写。如有需求,也可以采用
多文件编译的方式。主体部分的页码采用阿拉伯数字。

\begin{function}[updated=2018-01-15]{\footnote}
  \begin{fdusyntax}[deletetexcs={\footnote},
      morekeywords={\footnote}]
    \footnote(*\marg{脚注文字}*)
  \end{fdusyntax}
  插入脚注。脚注编号样式可利用 \opt{style/footnote-style} 选项控制,
  具体见 \ref{subsubsec:论文格式}~小节。
\end{function}

% TODO: \DescribeEnv{proof}
\begin{function}{axiom,corollary,definition,example,lemma,
  proof,theorem}
  \begin{fdusyntax}[emph={[2]proof}]
    \begin{proof}(*\oarg{小标题}*)
      (*\meta{证明过程}*)
    \end{proof}
  \end{fdusyntax}
  一系列预定义的数学环境。具体含义见表~\ref{tab:theorem}。
\end{function}

\begin{table}[ht]
  \caption{预定义的数学环境} \label{tab:theorem}
  \centering
  \begin{tabular}{cccccccc}
    \toprule
    \textbf{名称} &
      \env{axiom}   & \env{corollary} & \env{definition} &
      \env{example} & \env{lemma}     & \env{proof}      &
      \env{theorem} \\
    \midrule
    \textbf{含义} &
      公理 & 推论 & 定义 & 例 & 引理 & 证明 & 定理 \\
    \bottomrule
  \end{tabular}
\end{table}

证明环境(\env{proof})的最后会添加证毕符号“$\QED$”。要确保
该符号在正确的位置显示,您需要按照 \ref{subsec:编译方式}~节
中的有关说明编译\emph{两次}。

\begin{function}[updated=2017-12-12]{\newtheorem}
  \begin{fdusyntax}[deletetexcs={\newtheorem},
      morekeywords={\newtheorem,\newtheorem*}]
    \newtheorem(*\oarg{选项}\marg{环境名}\marg{标题}*)
    \newtheorem*(*\oarg{选项}\marg{环境名}\marg{标题}*)
    \begin(*\marg{环境名}\oarg{小标题}*)
      (*\meta{内容}*)
    \end(*\marg{环境名}*)
  \end{fdusyntax}
  声明新的定理类环境(数学环境)。带星号的版本表示不进行编号,
  并且会默认添加证毕符号“$\QED$”。声明后,即可同预定义的数学环境
  一样使用。
\end{function}

事实上,表~\ref{tab:theorem} 中预定义的环境正是通过以下方式定义的:
\begin{latexexample}[deletetexcs={\newtheorem},
    morekeywords={\newtheorem,\newtheorem*}]
  \newtheorem*{proof}{证明}
  \newtheorem{axiom}{公理}
  \newtheorem{corollary}{定理}
  ...
\end{latexexample}

与 \cs{fdusetup} 相同,\cs{newtheorem} 的可选参数 \meta{选项}
也为一组键值列表。可用的选项见下。注意您无需输入“|theorem/|”。

\begin{function}{theorem/style}
  \begin{fdusyntax}[emph={[1]style}]
    style = (*<(plain)|margin|change|\\
      XXXXXX\mbox{}~~~~~~~~break|marginbreak|changebreak>*)
  \end{fdusyntax}
  定理类环境的总体样式。
\end{function}

\begin{function}{theorem/header-font}
  \begin{fdusyntax}[emph={[1]header-font}]
    header-font = (*\marg{字体}*)
  \end{fdusyntax}
  定理头(即标题)的字体。中文模板默认为 \tn{sffamily},即无衬线体
  (黑体);英文模板默认为 |\bfseries\upshape|,即加粗直立体。
\end{function}

\begin{function}{theorem/body-font}
  \begin{fdusyntax}[emph={[1]body-font}]
    body-font = (*\marg{字体}*)
  \end{fdusyntax}
  定理内容的字体。中文模板默认为 \tn{fdu@kai},即楷体;英文模板
  默认为 \tn{itshape},即斜体。
\end{function}

\begin{function}{theorem/qed}
  \begin{fdusyntax}[emph={[1]qed}]
    qed = (*\marg{符号}*)
  \end{fdusyntax}
  定理结束标记(即证毕符号)。如果用 \cs{newtheorem} 声明定理,
  则默认为空;用 \cs{newtheorem*} 声明,则默认为
  |\ensuremath{\QED}|,即“$\QED$”。
\end{function}

\begin{function}{theorem/counter}
  \begin{fdusyntax}[emph={[1]counter}]
    counter = (*\marg{计数器}*)
  \end{fdusyntax}
  定理计数器,表示定理编号在 \meta{计数器} 的下一级,并会随
  \meta{计数器} 的变化而清零。\scite{刘海洋2013latex入门}
  默认为 |chapter|,表示按章编号。使用 \cs{newtheorem*} 时,
  该选项无效。
\end{function}

\begin{function}{\caption}
  \begin{fdusyntax}[deletetexcs={\caption},morekeywords={\caption}]
    \caption(*\marg{图表标题}*)
    \caption(*\oarg{短标题}\marg{长标题}*)
  \end{fdusyntax}
  插入图表标题。可选参数 \meta{短标题} 用于图表目录。在
  \meta{长标题} 中,您可以进行长达多段的叙述;但 \meta{短标题}
  和单独的 \meta{图表标题} 中则不允许分段。
  \scite{刘海洋2013latex入门}
\end{function}

按照排版惯例,建议您将表格的标题放置在绘制表格的命令之前,
而将图片的标题放置在绘图或插图的命令之后。另需注意,
\tn{caption} 命令必须放置在浮动体环境(如 \env{table} 和
\env{figure})中。

\subsubsection{豹尾}

\begin{function}{\backmatter}
  声明后置部分开始。
\end{function}

后置部分包含参考文献、声明页等。

\begin{function}[updated=2018-01-25]{\printbibliography}
  \begin{fdusyntax}[morekeywords={\printbibliography}]
    \printbibliography(*\oarg{选项}*)
  \end{fdusyntax}
  打印参考文献列表。如果 \kvopt{bib-backend}{bibtex},则 \meta{选项}
  无效,相当于 \tn{bibliography} \texttt{\marg{文献数据库}},其中的
  \meta{文献数据库} 可利用 \opt{style/bib-resource} 选项指定,具体见
  \ref{subsubsec:论文格式}~小节;而如果 \kvopt{bib-backend}^^A
  {biblatex},则该命令由 \pkg{biblatex} 宏包直接提供,可用选项请
  参阅其文档 \cite{biblatex}。
\end{function}

\section{宏包依赖情况}

使用不同编译方式、指定不同选项,会导致宏包依赖情况有所不同。
具体如下:
\begin{itemize}
  \item 在任何情况下,本模板都会\emph{显式}调用以下宏包
    (或文档类):
    \begin{itemize}
      \item \pkg{expl3}、\pkg{xparse}、\pkg{xtemplate} 和
        \pkg{l3keys2e},用于构建 \LaTeX3 编程环境
        \scite{source3}。它们分属 \pkg{l3kernel} 和
        \pkg{l3packages} 宏集。
      \item \cls{ctexbook},提供中文排版的通用框架。属于 \CTeX{}
        宏集 \scite{CTeX}。
      \item \pkg{amsmath},对 \LaTeX{} 的数学排版功能进行了
        全面扩展。属于 \AmSLaTeX{} 套件。
      \item \pkg{unicode-math},负责处理 Unicode 编码的
        OpenType 数学字体。
      \item \pkg{geometry},用于调整页面尺寸。
      \item \pkg{fancyhdr},处理页眉页脚。
      \item \pkg{footmisc},处理脚注。
      \item \pkg{ntheorem},提供增强版的定理类环境。
      \item \pkg{graphicx},提供图形插入的接口。
      \item \pkg{longtable},长表格(允许跨页)支持。
      \item \pkg{caption},用于设置题注。
      \item \pkg{xcolor},提供彩色支持。
      \item \pkg{hyperref},提供交叉引用、超链接、电子书签等功能。
    \end{itemize}
  \item 开启 \kvopt{style/footnote-style}{pifont} 后,会调用
    \pkg{pifont} 宏包。它属于 \pkg{psnfss} 套件。
  \item 开启 \kvopt{style/bib-backend}{bibtex} 后,会调用
    \pkg{natbib} 宏包,并依赖 \BibTeX{} 程序。参考文献样式由
    \pkg{gbt7714} 宏包提供 \scite{natbib,gbt7714}。
  \item 开启 \kvopt{style/bib-backend}{biblatex} 后,会调用
    \pkg{biblatex} 宏包,并依赖 \biber{} 程序。参考文献样式由
    \pkg{biblatex-gb7714-2015} 宏包提供
    \scite{biblatex,biblatex-gb7714-2015}。
\end{itemize}

这里只列出了本模板直接调用的宏包。这些宏包自身的调用情况,
此处不再具体展开。如有需要,请参阅相关文档。

\begin{thebibliography}{99}

\newcommand\urlprefix{\newline\hspace*{\fill}}
\let\OldUrl=\url
\renewcommand\url[2][]{{\small\textit{#1}~\OldUrl{#2}}}
\newcommand\CTANurl[2][]{{\small\textit{#1}~\href{http://mirror.ctan.org/#2}^^A
  {\ttfamily CTAN://#2}}}

\subsection{图书}

\bibitem{knuth1986texbook}
\textsc{Knuth D E}.
\newblock \textit{The \TeX book: Computers \& Typesetting, volume A} [M].
\newblock Boston: Addison--Wesley Publishing Company, 1986
\urlprefix \CTANurl[源代码^^A
  \footnote{此代码只可作为学习之用。未经 Knuth 本人同意,您不应当编译此文档。}:]^^A
  {systems/knuth/dist/tex/texbook.tex}

\bibitem{mittelbach2004latexcompanion}
\textsc{Mittelbach F} and \textsc{Goossens M}.
\newblock \textit{The \LaTeX{} Companion} [M].
\newblock 2nd ed.
\newblock Boston: Addison--Wesley Publishing Company, 2004

\bibitem{胡伟2017latex2e}
胡伟.
\newblock \textit{\LaTeXe{} 文类和宏包学习手册} [M].
\newblock 北京: 清华大学出版社, 2017

\bibitem{刘海洋2013latex入门}
刘海洋.
\newblock \textit{\LaTeX{} 入门} [M].
\newblock 北京: 电子工业出版社, 2013

\subsection{标准、规范}

\bibitem{gb-t-7713.1-2006}
国务院学位委员会办公室, 全国信息与文献标准化技术委员会.
\newblock \textit{学位论文编写规则: GB/T 7713.1--2006} [S].
\newblock 北京: 中国标准出版社, 2007

\bibitem{gb-t-7714-2015}
全国信息与文献标准化技术委员会.
\newblock \textit{信息与文献\quad 参考文献著录规则: GB/T 7714--2015} [S].
\newblock 北京: 中国标准出版社, 2015

\bibitem{gb-t-15834-2011}
教育部语言文字信息管理司.
\newblock \textit{标点符号用法: GB/T 15834--2011} [S/OL].
\newblock 北京: 中国标准出版社, 2012
\urlprefix\url{http://www.moe.gov.cn/ewebeditor/uploadfile/2015/01/13/20150113091548267.pdf}

\bibitem{clreq}
W3C.
\newblock \textit{中文排版需求(Requirements for Chinese Text Layout)} [EB/OL].
\newblock (2019-03-13) 
\urlprefix\url{https://w3c.github.io/clreq/}

\bibitem{复旦大学论文规范}
复旦大学图书馆, 复旦大学研究生院.
\newblock \textit{复旦大学博士、硕士学位论文规范} [EB/OL].
\newblock 2017 年 3 月修订版.
\newblock (2017-03-27) 
\urlprefix\url{http://www.gs.fudan.edu.cn/_upload/article/4c/a8/a82545ef443b9c057c14ba13782c/c883c6f3-6d7f-410c-8f30-d8bde6fcb990.doc}

\subsection{宏包、模版}

\bibitem{source2e}
\textsc{Braams J}, \textsc{Carlisle D}, \textsc{Jeffrey A}, et al.
\newblock \textit{The \LaTeXe{} Sources} [CP/OL].
\newblock (2018-12-01) 
\urlprefix\url{https://ctan.org/pkg/latex}
\urlprefix\CTANurl[源代码:]{macros/latex/base/source2e.pdf}

\bibitem{CTeX}
\textsc{CTEX.ORG}.
\newblock \textit{\CTeX{} 宏集手册} [EB/OL].
\newblock version 2.4.14,
\newblock (2018-05-02) 
\urlprefix\url{https://ctan.org/pkg/ctex}
\urlprefix\CTANurl[文档及源代码:]{language/chinese/ctex/ctex.pdf}

\bibitem{xeCJK}
\textsc{CTEX.ORG}.
\newblock \textit{\pkg{xeCJK} 宏包} [EB/OL].
\newblock version 3.7.1,
\newblock (2018-04-30) 
\urlprefix\url{https://ctan.org/pkg/xecjk}
\urlprefix\CTANurl[文档及源代码:]{macros/xetex/latex/xecjk/xeCJK.pdf}

\bibitem{natbib}
\textsc{Daly P W}.
\newblock \textit{Natural Sciences Citations and References} [EB/OL].
\newblock version 8.31b,
\newblock (2010-09-13) 
\urlprefix\url{https://ctan.org/pkg/natbib}
\urlprefix\CTANurl[文档及源代码:]{macros/latex/contrib/natbib/natbib.pdf}

\bibitem{source3}
\textsc{The \LaTeX3 Project}.
\newblock \textit{The \LaTeX3 Sources} [CP/OL].
\newblock (2019-03-05) 
\urlprefix\url{https://ctan.org/pkg/l3kernel}
\urlprefix\CTANurl[源代码:]{macros/latex/contrib/l3kernel/source3.pdf}

\bibitem{biblatex}
\textsc{Lehman P}, \textsc{Kime P}, \textsc{Boruvka A}, et al.
\newblock \textit{The \pkg{biblatex} Package} [EB/OL].
\newblock version 3.12,
\newblock (2018-10-18) 
\urlprefix\url{https://ctan.org/pkg/biblatex}
\urlprefix\CTANurl[文档:]{macros/latex/contrib/biblatex/doc/biblatex.pdf}

\bibitem{lshort}
\textsc{Oetiker T}, \textsc{Partl H}, \textsc{Hyna I}, et al.
\newblock \textit{The Not So Short Introduction to \LaTeXe{}: Or \LaTeXe{} in 139 minutes} [EB/OL].
\newblock version 6.2,
\newblock (2018-02-28) 
\urlprefix\url{https://ctan.org/pkg/lshort-english}
\urlprefix\CTANurl[文档:]{info/lshort/english/lshort.pdf}

\bibitem{lshort-zh-cn}
\textsc{Oetiker T}, \textsc{Partl H}, \textsc{Hyna I}, et al.
\newblock \textit{一份不太简短的 \LaTeXe{} 介绍: 或 106 分钟了解 \LaTeXe{}} [EB/OL].
\newblock \CTeX{} 开发小组, 译.
\newblock 原版版本 version 6.2, 中文版本 version 6.0,
\newblock (2018-09-01) 
\urlprefix\url{https://ctan.org/pkg/lshort-zh-cn}
\urlprefix\CTANurl[文档:]{info/lshort/chinese/lshort-zh-cn.pdf}

\bibitem{biblatex-gb7714-2015}
胡振震.
\newblock \textit{符合 GB/T 7714-2015 标准的 biblatex 参考文献样式} [EB/OL].
\newblock version 1.0q,
\newblock (2019-02-11) 
\urlprefix\url{https://ctan.org/pkg/biblatex-gb7714-2015}
\urlprefix\CTANurl[文档:]{biblatex-contrib/biblatex-gb7714-2015/biblatex-gb7714-2015.pdf}

\bibitem{gbt7714}
李泽平(\textsc{Zeping L}).
\newblock \textit{GB/T 7714-2015 \BibTeX{} Style} [EB/OL].
\newblock version 1.0.9,
\newblock (2018-08-05) 
\urlprefix\url{https://ctan.org/pkg/gbt7714}
\urlprefix\CTANurl[文档:]{biblio/bibtex/contrib/gbt7714/gbt7714.pdf}

\bibitem{cquthesis}
李振楠.
\newblock \textit{\textsc{CquThesis}:重庆大学毕业论文 \LaTeX{} 模板} [EB/OL].
\newblock version 1.30,
\newblock (2018-02-23) 
\urlprefix\url{https://ctan.org/pkg/cquthesis}
\urlprefix\CTANurl[文档及源代码:]{macros/latex/contrib/cquthesis/cquthesis.pdf}

\bibitem{thuthesis}
薛瑞尼.
\newblock \textit{\textsc{ThuThesis}:清华大学学位论文模板} [EB/OL].
\newblock version 5.4.5,
\newblock (2018-05-17) 
\urlprefix\url{https://ctan.org/pkg/thuthesis}
\urlprefix\CTANurl[文档及源代码:]{macros/latex/contrib/thuthesis/thuthesis.pdf}

\emph{以下模版未收录至 CTAN,但仍然保持活跃更新。}

\bibitem{sjtuthesis}
\textsc{SJTUG}.
\newblock \textit{上海交通大学 \XeLaTeX{} 学位论文及课程论文模板} [EB/OL].
\newblock version 0.10.2,
\newblock (2018-11-05)
\urlprefix\url{https://github.com/sjtug/SJTUThesis}

\bibitem{ustcthesis}
\textsc{USTC \TeX{} User Group}.
\newblock \textit{中国科学技术大学学位论文 \LaTeX{} 模板} [EB/OL].
\newblock version 3.1.03,
\newblock (2019-01-01)
\urlprefix\url{https://github.com/ustctug/ustcthesis}

\bibitem{ucasthesis}
\textsc{mohuangrui}.
\newblock \textit{\pkg{ucasthesis} 国科大学位论文 \LaTeX{} 模板} [EB/OL].
\newblock (2019-03-14)
\urlprefix\url{https://github.com/mohuangrui/ucasthesis}

\emph{以下模版现已停止更新。}

\bibitem{pandoxie2014fduthesislatex}
\textsc{Pandoxie}.
\newblock \textit{Fudan University-Latex Template} [EB/OL].
\newblock (2014-06-07) 
\urlprefix\url{https://github.com/Pandoxie/FDU-Thesis-Latex}

\bibitem{richard2016fudanthesis}
\textsc{richard}.
\newblock \textit{复旦大学硕士学位论文模板} [EB/OL].
\newblock (2016-01-31) 
\urlprefix\url{https://github.com/richarddzh/fudan-thesis}

\bibitem{数院毕业论文格式}
复旦大学数学科学学院.
\newblock \textit{毕业论文格式 tex 版和 word 版} [EB/OL].
\urlprefix\url{http://math.fudan.edu.cn/show.aspx?info_lb=664&flag=101&info_id=1816}

\bibitem{数院毕业论文格式更新}
复旦大学数学科学学院.
\newblock \textit{毕业论文格式: Word、\TeX{} 模板更新} [EB/OL].
\urlprefix\url{http://math.fudan.edu.cn/Show.aspx?info_lb=664&info_id=1855&flag=101}

\subsection{其他}

\bibitem{wright2009dtxfile}
\textsc{Wright J}.
\newblock \textit{A model dtx file} [EB/OL].
\newblock (2009-10-06) 
\urlprefix\url{https://www.texdev.net/2009/10/06/a-model-dtx-file/}

\bibitem{孔雀计划}
刘庆(\textsc{Eric Q L}).
\newblock \textit{孔雀计划:中文字体排印的思路} [EB/OL].
\urlprefix\url{https://thetype.com/kongque/}

\end{thebibliography}

\end{document}
