% \iffalse meta-comment
%
% Copyright (C) 2017--2019 by Xiangdong Zeng <xdzeng96@gmail.com>
%
% This work may be distributed and/or modified under the conditions of the
% LaTeX Project Public License, either version 1.3c of this license or (at
% your option) any later version. The latest version of this license is in:
%
%   http://www.latex-project.org/lppl.txt
%
% and version 1.3 or later is part of all distributions of LaTeX version
% 2005/12/01 or later.
%
% This work has the LPPL maintenance status `maintained'.
%
% The Current Maintainer of this work is Xiangdong Zeng.
%
% \fi
%
% \begin{implementation}
%
% \section{封面}
%
% \subsection{信息录入}
%
% \begin{variable}{\l_@@_info_title_tl,
%   \l_@@_info_date_tl,
%   \l_@@_info_author_tl,
%   \l_@@_info_supervisor_tl,
%   \l_@@_info_department_tl,
%   \l_@@_info_major_tl,
%   \l_@@_info_student_id_tl,
%   \l_@@_info_school_id_tl,
%   \l_@@_info_clc_tl,
%   \l_@@_info_instructors_clist,
%   \l_@@_info_keywords_clist}
% 封面所需的一些字段。
%    \begin{macrocode}
\clist_map_inline:nn
  {
    title, date, author, supervisor, department, major, student_id,
    school_id, clc
  }
  { \tl_new:c { l_@@_info_ #1 _tl } }
\clist_new:N \l_@@_info_instructors_clist
\clist_new:N \l_@@_info_keywords_clist
%    \end{macrocode}
% \end{variable}
%
% \begin{variable}{\l_@@_info_title_en_tl,
%   \l_@@_info_author_en_tl,
%   \l_@@_info_supervisor_en_tl,
%   \l_@@_info_department_en_tl,
%   \l_@@_info_major_en_tl,
%   \l_@@_info_keywords_en_clist}
% 对应的英文字段。
%    \begin{macrocode}
\clist_map_inline:nn
  { title, author, supervisor, department, major }
  { \tl_new:c { l_@@_info_ #1 _en_tl } }
\clist_new:N \l_@@_info_keywords_en_clist
%    \end{macrocode}
% \end{variable}
%
% \begin{variable}{\l_@@_info_degree_type_int}
% 学位类型。1 为学术学位,2 为专业学位。
%    \begin{macrocode}
\int_new:N \l_@@_info_degree_type_int
%    \end{macrocode}
% \end{variable}
%
% 定义 |fdu/info| 键值类。
%    \begin{macrocode}
\keys_define:nn { fdu / info }
  {
%    \end{macrocode}
%
% \changes{v0.7}{2018/02/01}{新增 \opt{info/degree} 选项。}
%
% \begin{macro}{info/degree}
% 学位类型。只对硕士论文有效。
%    \begin{macrocode}
    degree      .choices:nn  =
      { academic, professional }
      { \int_set_eq:NN \l_@@_info_degree_type_int \l_keys_choice_int },
%    \end{macrocode}
% \end{macro}
%
% \begin{macro}{info/title,info/title*}
% 论文题目。以下带星号的项目均表示相应的英文字段。
%    \begin{macrocode}
    title       .tl_set:N    = \l_@@_info_title_tl,
    title*      .tl_set:N    = \l_@@_info_title_en_tl,
%    \end{macrocode}
% \end{macro}
%
% \begin{macro}{info/date}
% 论文完成日期。
%    \begin{macrocode}
    date        .tl_set:N    = \l_@@_info_date_tl,
%    \end{macrocode}
% \end{macro}
%
% \begin{macro}{info/author,info/author*}
% 作者姓名。
%    \begin{macrocode}
    author      .tl_set:N    = \l_@@_info_author_tl,
    author*     .tl_set:N    = \l_@@_info_author_en_tl,
%    \end{macrocode}
% \end{macro}
%
% \begin{macro}{info/supervisor,info/supervisor*}
% 导师姓名。
%    \begin{macrocode}
    supervisor  .tl_set:N    = \l_@@_info_supervisor_tl,
%   supervisor* .tl_set:N    = \l_@@_info_supervisor_en_tl,
%    \end{macrocode}
% \end{macro}
%
% \begin{macro}{info/instructors}
% 指导小组成员。
%    \begin{macrocode}
    instructors .clist_set:N = \l_@@_info_instructors_clist,
%    \end{macrocode}
% \end{macro}
%
% \begin{macro}{info/department,info/department*}
% 院系。
%    \begin{macrocode}
    department  .tl_set:N    = \l_@@_info_department_tl,
%   department* .tl_set:N    = \l_@@_info_department_en_tl,
%    \end{macrocode}
% \end{macro}
%
% \begin{macro}{info/major,info/major*}
% 专业。
%    \begin{macrocode}
    major       .tl_set:N    = \l_@@_info_major_tl,
%   major*      .tl_set:N    = \l_@@_info_major_en_tl,
%    \end{macrocode}
% \end{macro}
%
% \begin{macro}{info/student-id}
% 学号。
%    \begin{macrocode}
    student-id  .tl_set:N    = \l_@@_info_student_id_tl,
%    \end{macrocode}
% \end{macro}
%
% \begin{macro}{info/school-id}
% 学校代码。
%    \begin{macrocode}
    school-id   .tl_set:N    = \l_@@_info_school_id_tl,
%    \end{macrocode}
% \end{macro}
%
% \begin{macro}{info/keywords,info/keywords*}
% 论文关键字。
%    \begin{macrocode}
    keywords    .clist_set:N = \l_@@_info_keywords_clist,
    keywords*   .clist_set:N = \l_@@_info_keywords_en_clist,
%    \end{macrocode}
% \end{macro}
%
% \begin{macro}{info/clc}
% 中图分类号。
%    \begin{macrocode}
    clc         .tl_set:N    = \l_@@_info_clc_tl
  }
%    \end{macrocode}
% \end{macro}
%
% \changes{v0.4}{2017/08/10}{新增 \opt{style/logo} 与
%   \opt{style/logo-size} 选项。}
%
% \begin{variable}{\l_@@_cover_logo_tl,\l_@@_cover_logo_size_clist}
%    \begin{macrocode}
\tl_new:N    \l_@@_cover_logo_tl
\clist_new:N \l_@@_cover_logo_size_clist
%    \end{macrocode}
% \end{variable}
%
% \begin{macro}{style/logo,style/logo-size}
% 校名图片的文件名和尺寸。
%    \begin{macrocode}
\keys_define:nn { fdu / style }
  {
    logo      .tl_set:N    = \l_@@_cover_logo_tl,
    logo-size .clist_set:N = \l_@@_cover_logo_size_clist
  }
%    \end{macrocode}
% \end{macro}
%
% \subsection{密级}
%
% \changes{v0.3}{2017/07/04}{新增 \opt{info/secret-level} 与
%   \opt{info/secret-year} 选项。}
%
% \begin{variable}{\l_@@_secret_bool}
% 是否显示密级。
%    \begin{macrocode}
\bool_new:N \l_@@_secret_bool
%    \end{macrocode}
% \end{variable}
%
% \begin{variable}{\l_@@_info_secret_level_tl}
% 保存当前的密级。
%    \begin{macrocode}
\tl_new:N \l_@@_info_secret_level_tl
%    \end{macrocode}
% \end{variable}
%
%    \begin{macrocode}
\keys_define:nn { fdu / info }
  {
%    \end{macrocode}
%
% \begin{macro}{info/secret-level}
% \changes{v0.6}{2017/11/11}{不再依赖 XITS-Math 字体。}
% 密级。\opt{none} 表示不涉密,\opt{i}、\opt{ii}、\opt{iii} 分别为
% 秘密、机密、绝密。
%    \begin{macrocode}
    secret-level .choices:nn  =
      { none, i, ii, iii }
      {
        \int_compare:nTF { \l_keys_choice_int >= 2 }
          {
            \bool_set_true:N \l_@@_secret_bool
            \tl_set:Nn \l_@@_info_secret_level_tl
              {
                \clist_item:Nn \c_@@_secret_clist
                  { \l_keys_choice_int - 1 }
              }
          }
          { \bool_set_false:N \l_@@_secret_bool }
      },
    secret-level .value_required:n = true,
%    \end{macrocode}
% \end{macro}
%
% \begin{macro}{info/secret-year}
% 保密年限。
%    \begin{macrocode}
    secret-year  .tl_set:N = \l_@@_info_secret_year_tl
  }
%    \end{macrocode}
% \end{macro}
%
% \subsection{定义内部函数}
%
% \begin{macro}{\@@_spread_box:nn}
% 分散对齐的水平盒子。
% \begin{arguments}
%   \item 宽度
%   \item 内容
% \end{arguments}
% 利用 \cs{tl_map_inline:nn} 在字符间插入 \tn{hfil};紧随其后的 \tn{unskip}
% 将会去掉最后一个 \tn{hfil}。见 \url{https://tex.stackexchange.com/q/169689}。
% |#2| 需要完全展开以避免 underfull 警告。
%    \begin{macrocode}
\cs_new_protected:Npn \@@_spread_box:nn #1#2
  {
    \mode_leave_vertical:
    \hbox_to_wd:nn {#1}
      { \tl_map_inline:xn {#2} { ##1 \hfil } \unskip }
  }
%    \end{macrocode}
% \end{macro}
%
% \begin{macro}{\@@_center_box:nn,\@@_center_box:Vn}
% 居中对齐的水平盒子。
%    \begin{macrocode}
\cs_new_protected:Npn \@@_center_box:nn #1#2
  {
    \mode_leave_vertical:
    \hbox_to_wd:nn {#1} { \hfil #2 \hfil }
  }
\cs_generate_variant:Nn \@@_center_box:nn  { Vn }
%    \end{macrocode}
% \end{macro}
%
% \begin{macro}{\@@_fixed_width_box:nn}
% 限宽盒子(允许换行)。
%    \begin{macrocode}
\cs_new:Npn \@@_fixed_width_box:nn #1#2
  { \parbox {#1} {#2} }
%    \end{macrocode}
% \end{macro}
%
% \begin{macro}{\@@_fixed_width_center_box:nn}
% 居中对齐的限宽盒子(允许换行)。
%    \begin{macrocode}
\cs_new:Npn \@@_fixed_width_center_box:nn #1#2
  { \parbox {#1} { \centering #2 } }
%    \end{macrocode}
% \end{macro}
%
% \begin{macro}{\@@_get_text_width:Nn,\@@_get_text_width:NV}
% 获取文本宽度,并存入 |dim| 型变量。
% \begin{arguments}
%   \item |dim| 型变量
%   \item 内容
% \end{arguments}
%    \begin{macrocode}
\cs_new:Npn \@@_get_text_width:Nn #1#2
  {
    \hbox_set:Nn \l_@@_tmpa_box {#2}
    \dim_set:Nn #1 { \box_wd:N \l_@@_tmpa_box }
  }
\cs_generate_variant:Nn \@@_get_text_width:Nn { NV }
%    \end{macrocode}
% \end{macro}
%
% \begin{macro}{\@@_get_max_text_width:NN}
% \changes{v0.6}{2017/11/24}{移除不必要的字号设置。}
% 获取多个文本中的最大宽度,并存入 |dim| 型变量。
% \begin{arguments}
%   \item |dim| 型变量
%   \item 文本 |clist|
% \end{arguments}
% 当 \cs{l_@@_tmpa_clist} 非空时,弹出最后一个元素
% 赋给 \cs{l_@@_tmpa_tl},获取其长度后与 |#1| 进行比较,
% 二者中较大的那一个将成为 |#1| 的新值。
% 不断循环,直至 \cs{l_@@_tmpa_clist} 为空。
%    \begin{macrocode}
\cs_new:Npn \@@_get_max_text_width:NN #1#2
  {
%    \end{macrocode}
% 这里用 |group| 确保局部变量不会被污染。
%    \begin{macrocode}
    \group_begin:
      \clist_set_eq:NN \l_@@_tmpa_clist #2
      \bool_until_do:nn { \clist_if_empty_p:N \l_@@_tmpa_clist }
        {
          \clist_pop:NN \l_@@_tmpa_clist \l_@@_tmpa_tl
          \@@_get_text_width:NV \l_@@_tmpa_dim \l_@@_tmpa_tl
          \dim_gset:Nn #1 { \dim_max:nn {#1} { \l_@@_tmpa_dim } }
        }
    \group_end:
  }
%    \end{macrocode}
% \end{macro}
%
% \begin{macro}{\@@_blank_underline:n}
% \changes{v0.4}{2017/08/14}{改用 \tn{rule} 绘制下划线,不再依赖
%   \pkg{ulem} 宏包。}
% 下划线占位符。|#1|: 长度。
%    \begin{macrocode}
\cs_new:Npn \@@_blank_underline:n #1
  { \rule [ -0.5 ex ] {#1} { 0.4 pt } }
%    \end{macrocode}
% \end{macro}
%
% \begin{macro}{\@@_line_spread:N,\@@_line_spread:n}
% 设置行距。|#1|: 行距倍数 |fp| 变量。
%    \begin{macrocode}
\cs_new:Npn \@@_line_spread:N #1
  { \linespread { \fp_use:N #1 } \selectfont }
\cs_new:Npn \@@_line_spread:n #1
  { \linespread {#1} \selectfont }
%    \end{macrocode}
% \end{macro}
%
% \subsection{封面各部件}
%
% \changes{v0.5}{2017/09/19}{使用 \pkg{expl3} 以及内部函数改写
%   封面,减少对 \LaTeXe{} 的依赖。}
%
% \begin{macro}{\@@_cover_id:,\@@_cover_id_aux:n}
% 右上角的学校代码和学号。
%    \begin{macrocode}
\cs_new_protected:Npn \@@_cover_id:
  {
    \@@_fixed_width_box:nn { 120 pt }
      {
        \bool_if:NT \l_@@_secret_bool
          {
            \group_begin:
              \sffamily
              \@@_cover_id_aux:n { secret_level }
              \c_@@_name_secret_star_tl
              \l_@@_info_secret_year_tl
            \group_end:
            \par
          }
        \@@_cover_id_aux:n { school_id  } \par
        \@@_cover_id_aux:n { student_id }
      }
%    \end{macrocode}
% 插入一个宽度为负的水平盒子以减少右侧边距。
%    \begin{macrocode}
    \hbox_to_wd:nn { -24 pt } { }
  }
\cs_new:Npn \@@_cover_id_aux:n #1
  {
    \tl_use:c { c_@@_name_ #1 _tl }
    \c_@@_fwid_colon_tl
    \tl_use:c { l_@@_info_ #1 _tl }
  }
%    \end{macrocode}
% \end{macro}
%
% \begin{macro}{\@@_cover_logo:}
% 插入校名图片。根据参数 \opt{width} 和 \opt{height} 是否为空依次
% 判断。\cs{l_@@_cover_logo_size_clist} 中超过两个的参数将被忽略。
%    \begin{macrocode}
\cs_new_protected:Npn \@@_cover_logo:
  {
    \clist_pop:NN   \l_@@_cover_logo_size_clist \l_@@_tmpa_tl
    \clist_pop:NNTF \l_@@_cover_logo_size_clist \l_@@_tmpb_tl
      {
        \tl_if_empty:NTF \l_@@_tmpa_tl
          { \includegraphics [ height = \l_@@_tmpb_tl ] }
          {
            \includegraphics
              [ width  = \l_@@_tmpa_tl, height = \l_@@_tmpb_tl ]
          }
      }
      { \includegraphics [ width = \l_@@_tmpa_tl ] }
    { \l_@@_cover_logo_tl }
  }
%    \end{macrocode}
% \end{macro}
%
% \begin{macro}{\@@_cover_type:}
% 论文类型。
%    \begin{macrocode}
\cs_new_protected:Npn \@@_cover_type:
  {
    \tl_set:Nx \l_@@_tmpa_tl
      {
        \clist_item:Nn \c_@@_thesis_type_clist
          { \g_@@_thesis_type_int }
      }
    \@@_spread_box:nn { 0.45 \textwidth } { \l_@@_tmpa_tl }
  }
%    \end{macrocode}
% \end{macro}
%
% \begin{macro}{\@@_cover_degree:}
% \changes{v0.7c}{2019/03/12}{允许博士学位论文使用学位类型。}
% 学位类型。
%    \begin{macrocode}
\cs_new_protected:Npn \@@_cover_degree:
  {
    \int_compare:nT { \g_@@_thesis_type_int != 3 }
      {
        \c_@@_fwid_left_paren_tl
        \clist_item:Nn \c_@@_degree_type_clist
          { \l_@@_info_degree_type_int }
        \c_@@_fwid_right_paren_tl
      }
  }
%    \end{macrocode}
% \end{macro}
%
% \begin{macro}{\@@_cover_info:}
% 信息栏。
%    \begin{macrocode}
\cs_new_protected:Npn \@@_cover_info:
  {
    \begin{minipage} [ c ] { \textwidth }
      \centering \zihao { 4 }
%    \end{macrocode}
% 读取左侧名称字段。
%    \begin{macrocode}
      \clist_set:Nx \l_@@_tmpa_clist
        {
          \c_@@_name_department_tl,
          \c_@@_name_major_tl,
          \c_@@_name_author_tl,
          \c_@@_name_supervisor_tl,
          \c_@@_name_date_tl,
        }
%    \end{macrocode}
% 设置信息栏右侧宽度。读取各字段,并将最宽者的宽度赋给
% \cs{l_@@_tmpb_dim}。
%    \begin{macrocode}
      \clist_set:Nx \l_@@_tmpb_clist
        {
          { \l_@@_info_department_tl },
          { \l_@@_info_major_tl      },
          { \l_@@_info_author_tl     },
          { \l_@@_info_supervisor_tl },
          { \l_@@_info_date_tl       }
        }
      \@@_get_max_text_width:NN \l_@@_tmpb_dim \l_@@_tmpb_clist
%    \end{macrocode}
% 用循环输出各字段。
%    \begin{macrocode}
      \bool_until_do:nn
        { \clist_if_empty_p:N \l_@@_tmpa_clist }
        {
          \clist_pop:NN \l_@@_tmpa_clist \l_@@_tmpa_tl
          \clist_pop:NN \l_@@_tmpb_clist \l_@@_tmpb_tl
          \@@_spread_box:nn { 6 em } { \l_@@_tmpa_tl }
          \c_@@_fwid_colon_tl
          \@@_center_box:Vn \l_@@_tmpb_dim { \l_@@_tmpb_tl }
          \skip_vertical:n { 1 ex }
        }
    \end{minipage}
  }
%    \end{macrocode}
% \end{macro}
%
% \begin{macro}{\@@_cover_signature:N}
% 签名行。
%    \begin{macrocode}
\cs_new_protected:Npn \@@_cover_signature:N #1
  {
    \clist_map_inline:Nn #1
      {
        ##1 \c_@@_fwid_colon_tl
        \@@_blank_underline:n { 6 em }
        \@@_quad:
      }
  }
%    \end{macrocode}
% \end{macro}
%
% \subsection{封面模板}
%
% \changes{v0.7}{2018/02/27}{使用 \pkg{xtemplate} 重构封面布局。}
%
% 声明封面对象。不需要带参数。
%    \begin{macrocode}
%<@@=fdu_cover>
\DeclareObjectType { fdu / cover } { \c_zero_int }
%    \end{macrocode}
%
% \begin{macro}{\DeclareCoverTemplate,\fdu_cover_declare_template:nn}
% 声明封面模板。
% \begin{arguments}
%   \item 模板名称
%   \item 封面部件列表,以逗号分隔
% \end{arguments}
%    \begin{macrocode}
\NewDocumentCommand \DeclareCoverTemplate { m m }
  { \fdu_cover_declare_template:nn {#1} {#2} }
\cs_new_protected:Npn \fdu_cover_declare_template:nn #1#2
  {
    \tl_set:Nn \l_@@_template_tl {#1}
%    \end{macrocode}
% 构建模板接口。
%    \begin{macrocode}
    \@@_declare_template_interface:nx {#1}
      {
        format      : tokenlist,
        top-skip    : skip,
        bottom-skip : skip,
        \clist_map_function:nN {#2} \@@_key_type:n
      }
%    \end{macrocode}
% 声明所用变量。
%    \begin{macrocode}
    \tl_new:c   { l_@@ / #1 / format_tl   }
    \skip_new:c { l_@@ / #1 / top_skip    }
    \skip_new:c { l_@@ / #1 / bottom_skip }
    \clist_map_inline:nn {#2}
      {
        \tl_new:c   { l_@@ / #1 / ##1 / content_tl  }
        \tl_new:c   { l_@@ / #1 / ##1 / format_tl   }
        \skip_new:c { l_@@ / #1 / ##1 / bottom_skip }
      }
%    \end{macrocode}
% 声明模板代码。^^A 以下名字空间为 `fdu_cover' 而非 `fdu'
%    \begin{macrocode}
    \@@_declare_template_code:nxn {#1}
      {
        format      = \exp_not:c { l_@@ / #1 / format_tl   },
        top-skip    = \use:c     { l_@@ / #1 / top_skip    },
        bottom-skip = \use:c     { l_@@ / #1 / bottom_skip },
        \clist_map_function:nN {#2} \@@_key_binding:n
      }
      {
        \AssignTemplateKeys
        \tl_use:c       { l_@@ / #1 / format_tl }
        \__fdu_vspace:c { l_@@ / #1 / top_skip  }
        \clist_map_inline:nn {#2}
          {
            \use:c { @@ / #1 / ####1 / align:n }
              {
                \tl_use:c { l_@@ / #1 / ####1 / format_tl  }
                \tl_use:c { l_@@ / #1 / ####1 / content_tl }
                \par
              }
            \__fdu_vspace:c { l_@@ / #1 / ####1 / bottom_skip }
          }
        \__fdu_vspace:c { l_@@ / #1 / bottom_skip }
      }
  }
%    \end{macrocode}
% \end{macro}
%
% \begin{variable}{\l_@@_template_tl}
% 保存模板名称。
%    \begin{macrocode}
\tl_new:N \l_@@_template_tl
%    \end{macrocode}
% \end{variable}
%
% \begin{macro}{\@@_declare_template_interface:nn,
%   \@@_declare_template_code:nnn,
%   \@@_declare_template_interface:nx,
%   \@@_declare_template_code:nxn}
% 为了展开的方便,这里需要封装 \pkg{xtemplate} 的一些函数。
%    \begin{macrocode}
\cs_new_protected:Npn \@@_declare_template_interface:nn #1#2
  { \DeclareTemplateInterface { fdu / cover } {#1} { \c_zero_int } {#2} }
\cs_new_protected:Npn \@@_declare_template_code:nnn #1#2#3
  { \DeclareTemplateCode { fdu / cover } {#1} { \c_zero_int } {#2} {#3} }
\cs_generate_variant:Nn \@@_declare_template_interface:nn { nx  }
\cs_generate_variant:Nn \@@_declare_template_code:nnn     { nxn }
%    \end{macrocode}
% \end{macro}
%
% \begin{macro}{\@@_key_type:n}
%    \begin{macrocode}
\cs_new:Npn \@@_key_type:n #1
  {
    #1 / content     : tokenlist,
    #1 / format      : tokenlist,
    #1 / bottom-skip : skip,
    #1 / align       : choice { left, right, center, normal } = normal,
  }
%    \end{macrocode}
% \end{macro}
%
% \begin{macro}{\@@_key_binding:n}
%    \begin{macrocode}
\cs_new:Npn \@@_key_binding:n #1
  {
    #1 / content     =
      \exp_not:c
        { l_@@ / \l_@@_template_tl / #1 / content_tl  },
    #1 / format      =
      \exp_not:c
        { l_@@ / \l_@@_template_tl / #1 / format_tl   },
    #1 / bottom-skip =
      \exp_not:c
        { l_@@ / \l_@@_template_tl / #1 / bottom_skip },
    #1 / align       =
      {
        left   =
          \exp_not:N \cs_set_protected:cpn
            { @@ / \l_@@_template_tl / #1 / align:n }
            \exp_not:n {##1}
            {
              \exp_not:n
                {
                  \group_begin:
                    \flushleft ##1 \endflushleft
                  \group_end:
                }
            },
        right  =
          \exp_not:N \cs_set_protected:cpn
            { @@ / \l_@@_template_tl / #1 / align:n }
            \exp_not:n {##1}
            {
              \exp_not:n
                {
                  \group_begin:
                    \flushright ##1 \endflushright
                  \group_end:
                }
            },
        center =
          \exp_not:N \cs_set_protected:cpn
            { @@ / \l_@@_template_tl / #1 / align:n }
            \exp_not:n {##1}
            {
              \exp_not:n
                {
                  \group_begin:
                    \center ##1 \endcenter
                  \group_end:
                }
            },
        normal =
          \exp_not:N \cs_set_protected:cpn
            { @@ / \l_@@_template_tl / #1 / align:n }
            \exp_not:n {##1}
            { \exp_not:n { \group_begin: ##1 \group_end: } }
      },
  }
%<@@=fdu>
%    \end{macrocode}
% \end{macro}
%
% \subsection{绘制封面}
%
% \begin{macro}{\makecoveri,\makecoverii,\makecoveriii}
% 使用实例(instance)构建封一、封二、封三。
%    \begin{macrocode}
\NewDocumentCommand \makecoveri { }
  {
    \thispagestyle { empty }
    \UseInstance { fdu / cover } { cover-i-default }
  }
\NewDocumentCommand \makecoverii { }
  {
    \thispagestyle { empty }
    \UseInstance { fdu / cover } { cover-ii-default }
  }
\NewDocumentCommand \makecoveriii { }
  {
    \cleardoublepage
    \thispagestyle { empty }
    \UseInstance { fdu / cover } { cover-iii-default }
  }
%    \end{macrocode}
% \end{macro}
%
% 声明各封面模板组成部分。
%    \begin{macrocode}
\DeclareCoverTemplate { cover-i   }
  { id, logo, type, degree, title, title-en, info }
\DeclareCoverTemplate { cover-ii  } { title, name-list }
\DeclareCoverTemplate { cover-iii }
  {
    originality-decl-name,
    originality-decl-text,
    originality-decl-sig,
    authorization-decl-name,
    authorization-decl-text,
    authorization-decl-sig
  }
%    \end{macrocode}
%
% \changes{v0.7d}{2019/03/29}{封面中文标题改为加粗宋体(可能使用伪粗)。}
%
% 定义封面的具体配置参数。
%    \begin{macrocode}
\DeclareInstance { fdu / cover } { cover-i-default } { cover-i }
  {
%<class-en>    format                 = \@@_line_spread:N \c_@@_line_spread_fp,
    bottom-skip            = 0 pt plus 1.5 fill,
    id       / content     = \@@_cover_id:,
    logo     / content     = \@@_cover_logo:,
    type     / content     = \@@_cover_type:,
    degree   / content     = \@@_cover_degree:,
    title    / content     =
      \@@_fixed_width_center_box:nn
        { 0.9 \textwidth } { \l_@@_info_title_tl },
    title-en / content     =
      \@@_fixed_width_center_box:nn
        { 0.9 \textwidth } { \l_@@_info_title_en_tl },
    info     / content     = \@@_cover_info:,
    id       / format      = \zihao { -5 },
    type     / format      = \zihao {  2 },
    degree   / format      = \zihao {  4 },
    title    / format      = \zihao { -2 } \bfseries,
    title-en / format      = \@@_line_spread:n { 1.2 } \zihao { 4 } \bfseries,
    id       / bottom-skip = 0 pt plus 1.6 fill,
    logo     / bottom-skip = 0 pt plus 0.3 fill,
    type     / bottom-skip = -18 pt,
    degree   / bottom-skip = 0 pt plus 0.8 fill,
    title-en / bottom-skip = 0 pt plus 2.5 fill,
    id       / align       = right,
    logo     / align       = center,
    type     / align       = center,
    degree   / align       = center,
    title    / align       = center,
    title-en / align       = center,
    info     / align       = center,
  }
\DeclareInstance { fdu / cover } { cover-ii-default } { cover-ii }
  {
%<class-en>    format              = \@@_line_spread:N \c_@@_line_spread_fp,
    title     / content =
      \@@_spread_box:nn { 7 em } { \c_@@_name_instructors_tl },
    name-list / content =
      \clist_use:Nn \l_@@_info_instructors_clist { \par },
    title     / format  = \zihao { 2 } \sffamily,
    name-list / format  = \large,
    title     / align   = center,
    name-list / align   = center,
  }
\DeclareInstance { fdu / cover } { cover-iii-default } { cover-iii }
  {
    format                                =
%<class>      \@@_line_spread:n { 1.8 },
%<class-en>      \@@_line_spread:n { 1.8 } \dim_set:Nn \parindent { 2 \ccwd },
    top-skip                              = 0 pt plus 0.2 fill,
    bottom-skip                           = 0 pt plus 2.5 fill,
    originality-decl-name   / content     = \c_@@_name_orig_decl_tl,
    originality-decl-text   / content     = \c_@@_orig_decl_text_tl,
    originality-decl-sig    / content     =
      \@@_cover_signature:N \c_@@_orig_decl_sign_clist,
    authorization-decl-name / content     = \c_@@_name_auth_decl_tl,
    authorization-decl-text / content     = \c_@@_auth_decl_text_tl,
    authorization-decl-sig  / content     =
      \@@_cover_signature:N \c_@@_auth_decl_sign_clist,
    originality-decl-name   / format      =
      \@@_line_spread:n { 1.2 } \zihao { -2 } \bfseries,
    authorization-decl-name / format      =
      \@@_line_spread:n { 1.2 } \zihao { -2 } \bfseries,
    originality-decl-name   / bottom-skip = 0.4 cm,
    originality-decl-text   / bottom-skip = 0.4 cm,
    originality-decl-sig    / bottom-skip = 0 pt plus 2.5 fill,
    authorization-decl-name / bottom-skip = 0.4 cm,
    authorization-decl-text / bottom-skip = 0.4 cm,
    originality-decl-name   / align       = center,
    originality-decl-sig    / align       = right,
    authorization-decl-name / align       = center,
    authorization-decl-sig  / align       = right,
  }
%    \end{macrocode}
%
% \begin{macro}{style/auto-make-cover}
%^^A \begin{variable}{\l_@@_auto_make_cover_bool}
% 是否自动生成封面。
%    \begin{macrocode}
\bool_new:N \l_@@_auto_make_cover_bool
\keys_define:nn { fdu / style }
  {
    auto-make-cover .bool_set:N = \l_@@_auto_make_cover_bool,
    auto-make-cover .default:n  = true
  }
%    \end{macrocode}
%^^A \end{variable}
% \end{macro}
%
% 在 \env{document} 开始位置添加封面以及指导小组成员名单。
%    \begin{macrocode}
\AtBeginDocument
  {
    \bool_if:NT \l_@@_auto_make_cover_bool
      {
        \begin{titlepage}
          \makecoveri \newpage \makecoverii
        \end{titlepage}
      }
  }
%    \end{macrocode}
%
% 在 \env{document} 结束位置添加声明页。
%    \begin{macrocode}
\AtEndDocument
  { \bool_if:NT \l_@@_auto_make_cover_bool { \makecoveriii } }
%    \end{macrocode}
%
% \end{implementation}
%
