% \iffalse meta-comment
%
% Copyright (C) 2017--2019 by Xiangdong Zeng <xdzeng96@gmail.com>
%
% This work may be distributed and/or modified under the conditions of the
% LaTeX Project Public License, either version 1.3c of this license or (at
% your option) any later version. The latest version of this license is in:
%
%   http://www.latex-project.org/lppl.txt
%
% and version 1.3 or later is part of all distributions of LaTeX version
% 2005/12/01 or later.
%
% This work has the LPPL maintenance status `maintained'.
%
% The Current Maintainer of this work is Xiangdong Zeng.
%
% \fi
%
% \begin{implementation}
%
% \section{页面布局}
%
% 利用 \pkg{geometry} 宏包设置纸张大小、页面边距以及页眉高度。这里,
% $\SI{2.54}{\centi\meter}=\SI{1}{in}$,
% $\SI{3.18}{\centi\meter}=\SI{1.25}{in}$。
%    \begin{macrocode}
\geometry
  {
    paper      = a4paper,
    vmargin    = 2.54 cm,
    hmargin    = 3.18 cm,
    headheight = 15 pt
  }
%    \end{macrocode}
%
% 草稿模式下显示页面边框及页眉、页脚线 。
%    \begin{macrocode}
\bool_if:NT \g_@@_draft_bool { \geometry { showframe } }
%    \end{macrocode}
%
% \section{页眉页脚}
%
% 清除默认页眉页脚格式。
%    \begin{macrocode}
\fancyhf { }
%    \end{macrocode}
%
% \begin{variable}{\l_@@_header_center_mark_tl}
% 保存中间页眉的文字。正文中设置为空,目录、摘要、符号表等设置为相应标题。
%    \begin{macrocode}
\tl_new:N \l_@@_header_center_mark_tl
%    \end{macrocode}
% \end{variable}
%
% 构建页眉,要在单面或双面下分别设置。
%
% \cs{fancyhead} 的选项中,\opt{E} 和 \opt{O} 分别表示偶数(even)
% 和奇数(odd), 而 \opt{L}、\opt{R} 和 \opt{C} 则分别表示左
% (left)、右(right)和中间(center)。按照通常的排版规则,
% 在双面模式下,偶数页的中间页眉文字在左,奇数页则在右。单面模式下,
% 左右页眉都要显示。
%    \begin{macrocode}
\bool_if:NTF \g_@@_twoside_bool
%<*class>
  {
    \fancyhead [ EL ] { \small \nouppercase { \fdu@kai \leftmark  } }
    \fancyhead [ OR ] { \small \nouppercase { \fdu@kai \rightmark } }
  }
  {
    \fancyhead [ L ] { \small \nouppercase { \fdu@kai \leftmark  } }
    \fancyhead [ R ] { \small \nouppercase { \fdu@kai \rightmark } }
    \fancyhead [ C ]
      {
        \small \nouppercase
          { \fdu@kai \l_@@_header_center_mark_tl }
      }
  }
%</class>
%<*class-en>
  {
    \fancyhead [ EL ] { \small \nouppercase { \itshape \leftmark  } }
    \fancyhead [ OR ] { \small \nouppercase { \itshape \rightmark } }
  }
  {
    \fancyhead [ L ] { \small \nouppercase { \itshape \leftmark  } }
    \fancyhead [ R ] { \small \nouppercase { \itshape \rightmark } }
    \fancyhead [ C ]
      {
        \small \nouppercase
          { \itshape \l_@@_header_center_mark_tl }
      }
  }
%</class-en>
%    \end{macrocode}
%
% 构建页脚,用来显示页码。选项 \opt{C} 表示居中(center)。
%    \begin{macrocode}
\fancyfoot [ C ] { \small \thepage }
%    \end{macrocode}
%
% 关闭横线显示(未启用)。
%    \begin{macrocode}
% \RenewDocumentCommand \headrulewidth { } { 0 pt }
%    \end{macrocode}
%
% \begin{macro}{\cleardoublepage}
% 重定义 \tn{cleardoublepage},使得偶数页面在没有内容时也不显示页眉页脚,见
% \url{https://tex.stackexchange.com/a/1683}。最后清空中间页眉,确保正文部分
% 页眉显示正确。
%    \begin{macrocode}
\RenewDocumentCommand \cleardoublepage { }
  {
    \clearpage
    \bool_if:NT \g_@@_twoside_bool
      {
        \int_if_odd:nF \c@page
          { \hbox:n { } \thispagestyle { empty } \newpage }
      }
    \tl_gset:Nn \l_@@_header_center_mark_tl { }
  }
%    \end{macrocode}
% \end{macro}
%
% \pkg{ctex} 宏包使用 \opt{heading} 选项后,会把页面格式设置为 |headings|。
% 因此必须在 \pkg{ctex} 调用之后重新设置 \cs{pagestyle} 为 |fancy|。
%    \begin{macrocode}
\pagestyle { fancy }
%    \end{macrocode}
%
% \section{章节标题结构}
%
% |\keys_set:nn{ctex}| 实际相当于 \cs{ctexset}。
%    \begin{macrocode}
\keys_set:nn { ctex }
  {
%    \end{macrocode}
% 设置章(chapter)、节(section)与小节(sub-section)标题样式。
% 此处使用 \kvopt{fixskip}{true} 选项来抑制前后的多余间距。
%    \begin{macrocode}
    chapter =
      {
%<class>        format      = \huge \normalfont \sffamily \centering,
%<*class-en>
        format      = \centering,
        nameformat  = \LARGE \bfseries,
        titleformat = \huge \bfseries,
        aftername   = \par \nobreak \vskip 10 pt,
%</class-en>
        beforeskip  = 50 pt,
        afterskip   = 40 pt,
        number      = \@@_arabic:n { chapter },
        fixskip     = true
      },
    section =
      {
%<class>        format      = \Large \normalfont \sffamily \raggedright,
%<class-en>        format      = \Large \bfseries \raggedright,
        beforeskip  = 3.5 ex plus 1.0 ex minus 0.2 ex,
        afterskip   = 2.7 ex plus 0.5 ex,
        fixskip     = true
      },
    subsection =
      {
%<class>        format      = \large \normalfont \sffamily \raggedright,
%<class-en>        format      = \large \bfseries \raggedright,
        beforeskip  = 3.25 ex plus 1.0 ex minus 0.2 ex,
        afterskip   = 2.5  ex plus 0.3 ex,
        fixskip     = true
      }
  }
%    \end{macrocode}
%
% \changes{v0.7d}{2019/03/24}{优化目录、摘要、参考文献等的标题实现。}
%
% \begin{macro}{\@@_chapter:n,\@@_chapter:V}
% 手动生成章的标题,用于摘要、参考文献等。
%    \begin{macrocode}
\cs_new_protected:Npn \@@_chapter:n #1
  {
    \group_begin:
      \ctexset { chapter / numbering = false }
      \chapter {#1}
      \@@_chapter_header:n {#1}
    \group_end:
  }
\cs_generate_variant:Nn \@@_chapter:n { V }
%    \end{macrocode}
% \end{macro}
%
% \begin{macro}{\@@_chapter_no_toc:n,\@@_chapter_no_toc:V}
% 目录自身不出现在目录中,需特别处理。参考
% \url{https://tex.stackexchange.com/a/1821}。
%    \begin{macrocode}
\cs_new_protected:Npn \@@_chapter_no_toc:n #1
  {
    \chapter *           {#1}
    \@@_chapter_header:n {#1}
    \pdfbookmark [0] {#1} { toc }
  }
\cs_generate_variant:Nn \@@_chapter_no_toc:n { V }
%    \end{macrocode}
% \end{macro}
%
% \begin{macro}{\@@_chapter_header:n}
% 单页模式下,目录、摘要、符号表等的页眉中间为相应标题,左右为空。
%    \begin{macrocode}
\cs_new_protected:Npn \@@_chapter_header:n #1
  {
    \bool_if:NTF \g_@@_twoside_bool
      { \markboth {#1} {#1} }
      {
        \markboth { } { }
        \tl_gset:Nn \l_@@_header_center_mark_tl {#1}
      }
  }
%    \end{macrocode}
% \end{macro}
%
% \section{图表绘制;浮动体}
%
% \changes{v0.3}{2017/07/09}{支持浮动体。}
%
% 分别设置浮动体 \env{figure} 和 \env{table} 的标题样式。
%    \begin{macrocode}
\captionsetup [ figure ]
  {
    font     = small,
    labelsep = quad
  }
\captionsetup [ table  ]
  {
    font     = { small, sf },
    labelsep = quad
  }
%    \end{macrocode}
%
% \begin{macro}{\thefigure,\thetable}
% \changes{v0.7}{2018/01/17}{改为可完全展开的命令。}
% 重定义图表编号。
%    \begin{macrocode}
\cs_set:Npn \thefigure
  { \@@_arabic:n { chapter } - \@@_arabic:n { figure } }
\cs_set:Npn \thetable
  { \@@_arabic:n { chapter } - \@@_arabic:n { table  } }
%    \end{macrocode}
% \end{macro}
%
% \end{implementation}
%
