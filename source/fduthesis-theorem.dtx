% \iffalse meta-comment
%
% Copyright (C) 2017--2019 by Xiangdong Zeng <xdzeng96@gmail.com>
%
% This work may be distributed and/or modified under the conditions of the
% LaTeX Project Public License, either version 1.3c of this license or (at
% your option) any later version. The latest version of this license is in:
%
%   http://www.latex-project.org/lppl.txt
%
% and version 1.3 or later is part of all distributions of LaTeX version
% 2005/12/01 or later.
%
% This work has the LPPL maintenance status `maintained'.
%
% The Current Maintainer of this work is Xiangdong Zeng.
%
% \fi
%
% \begin{implementation}
%
% \section{定理环境}
%
% \changes{v0.3}{2017/05/07}{新增定理环境。}
%
% \begin{variable}{\c_@@_thm_style_plain_clist,
%   \c_@@_thm_style_break_clist}
% 保存 \opt{plain}、\opt{break} 两种类型的定理样式名称。
%    \begin{macrocode}
\clist_const:Nn \c_@@_thm_style_plain_clist
  { plain, margin, change }
\clist_const:Nn \c_@@_thm_style_break_clist
  { break, marginbreak, changebreak }
%    \end{macrocode}
% \end{variable}
%
% \begin{variable}{\l_@@_thm_style_tl,
%   \l_@@_thm_header_font_tl,
%   \l_@@_thm_body_font_tl,
%   \l_@@_thm_qed_tl,
%   \l_@@_thm_counter_tl}
% 定理所需的一些字段。
%    \begin{macrocode}
\tl_new:N \l_@@_thm_style_tl
\tl_new:N \l_@@_thm_header_font_tl
\tl_new:N \l_@@_thm_body_font_tl
\tl_new:N \l_@@_thm_qed_tl
\tl_new:N \l_@@_thm_counter_tl
%    \end{macrocode}
% \end{variable}
%
% \begin{macro}{theorem/style,
%   theorem/header-font,
%   theorem/body-font,
%   theorem/qed,
%   theorem/counter}
% 定义 |fdu/theorem| 键值类。
%    \begin{macrocode}
\keys_define:nn { fdu / theorem }
  {
    style       .tl_set:N  = \l_@@_thm_style_tl,
    header-font .tl_set:N  = \l_@@_thm_header_font_tl,
    body-font   .tl_set:N  = \l_@@_thm_body_font_tl,
    qed         .tl_set:N  = \l_@@_thm_qed_tl,
    counter     .tl_set:N  = \l_@@_thm_counter_tl
  }
%    \end{macrocode}
% \end{macro}
%
% \begin{macro}{\@@_thm_ntheorem_style:n,\@@_thm_ntheorem_new:w}
% 拷贝 \pkg{ntheorem} 命令。
%    \begin{macrocode}
\cs_new_eq:NN \@@_thm_ntheorem_style:n \theoremstyle
\cs_new_eq:NN \@@_thm_ntheorem_new:w   \newtheorem
%    \end{macrocode}
% \end{macro}
%
% \changes{v0.7}{2017/12/12}{原 \cs{fdunewtheorem} 命令更名为
%   \cs{new\-the\-orem}。}
%
% \begin{macro}{\newtheorem}
% 定义新的定理环境。
%    \begin{macrocode}
\RenewDocumentCommand \newtheorem { s o m m }
  {
%    \end{macrocode}
% 默认情况下,由 \cs{newtheorem*} 创建的定理其证毕符号为 \cs{QED},
% 而由 \cs{new\-the\-orem} 创建的则不带证毕符号。符号 \cs{QED} 由
% \pkg{unicode-math} 宏包提供。
%    \begin{macrocode}
    \IfBooleanTF {#1}
      { \tl_set:Nn \l_@@_thm_qed_tl { \ensuremath { \QED } } }
      { \tl_set:Nn \l_@@_thm_qed_tl { } }
%    \end{macrocode}
% 设置默认样式为 \opt{plain}。
%    \begin{macrocode}
% TODO: (2017-12-07) Move to interface
    \tl_set:Nn \l_@@_thm_style_tl { plain }
%    \end{macrocode}
% 处理可选参数。利用 |fdu/theorem| 键值对设置,并按此修改证毕符号、
% 定理头字体和定理正文字体。
%    \begin{macrocode}
    \IfValueT {#2} { \keys_set:nn { fdu / theorem } {#2} }
    \fdu_thm_set_header_font:V \l_@@_thm_header_font_tl
    \fdu_thm_set_body_font:V   \l_@@_thm_body_font_tl
    \fdu_thm_set_qed:V         \l_@@_thm_qed_tl
%    \end{macrocode}
% \cs{newtheorem} 负责创建编号定理,而 \cs{newtheorem*}
% 则负责创建无编号定理。以下分这两种情况处理。
%    \begin{macrocode}
    \IfBooleanTF {#1}
      {
%    \end{macrocode}
% 带 |*| 的版本原则上只接受 \opt{plain} 和 \opt{break} 两种样式,
% 其余样式将被转换成这两者其中之一。
%    \begin{macrocode}
        \clist_if_in:nVF { plain, break } \l_@@_thm_style_tl
          {
            \clist_if_in:NVTF
              \c_@@_thm_style_plain_clist \l_@@_thm_style_tl
              { \@@_thm_redefine_style:n { plain } }
              {
                \clist_if_in:NVTF
                  \c_@@_thm_style_break_clist \l_@@_thm_style_tl
                  { \@@_thm_redefine_style:n { break } }
                  {
                    \@@_error:nx { unknown-theorem-style }
                      { \l_@@_thm_style_tl }
                  }
              }
          }
%    \end{macrocode}
% \pkg{ntheorem} 宏包提供的无编号定理带有 |nonumber| 前缀,
% 这里将其加上。
%    \begin{macrocode}
        \tl_put_left:Nn \l_@@_thm_style_tl { nonumber }
        \fdu_thm_new_no_number:Vxx \l_@@_thm_style_tl {#3} {#4}
      }
      {
%    \end{macrocode}
% 不带 |*| 的版本支持不含“|nonumber|”的所有定理样式。
%    \begin{macrocode}
        \clist_clear:N \l_@@_tmpa_clist
        \clist_concat:NNN \l_@@_tmpa_clist
          \c_@@_thm_style_plain_clist \c_@@_thm_style_break_clist
        \clist_if_in:NVF \l_@@_tmpa_clist \l_@@_thm_style_tl
          {
            \@@_error:nx { unknown-theorem-style }
              { \l_@@_thm_style_tl }
          }
        \fdu_thm_new:VVxx \l_@@_thm_style_tl \l_@@_thm_counter_tl
          {#3} {#4}
      }
  }
%    \end{macrocode}
% \end{macro}
%
% \begin{macro}{\@@_thm_redefine_style:n}
% 重定义定理样式,并给出警告。
%    \begin{macrocode}
\cs_new:Npn \@@_thm_redefine_style:n #1
  {
    \@@_warning:nxx { redefine-theorem-style }
      {#1} { \l_@@_thm_style_tl }
    \tl_set:Nn \l_@@_thm_style_tl {#1}
  }
%    \end{macrocode}
% \end{macro}
%
% 提示信息。
%    \begin{macrocode}
\@@_msg_new:nn { redefine-theorem-style }
  { Theorem~ style~ "#2"~ will~ be~ redefined~ as~ "#1". }
\@@_msg_new:nn { unknown-theorem-style }
  { Theorem~ style~ "#1"~ is~ unknown. }
%    \end{macrocode}
%
% \begin{macro}{\fdu_thm_new:nnnn,\fdu_thm_new:VVxx}
% 带编号的定理环境。
% \begin{arguments}
%   \item 样式
%   \item 计数器
%   \item 定理环境名称
%   \item 定理头文字
% \end{arguments}
%    \begin{macrocode}
\cs_new:Npn \fdu_thm_new:nnnn #1#2#3#4
  {
    \@@_thm_ntheorem_style:n {#1}
    \@@_thm_ntheorem_new:w   {#3} {#4} [#2]
  }
\cs_generate_variant:Nn \fdu_thm_new:nnnn { VVxx }
%    \end{macrocode}
% \end{macro}
%
% \begin{macro}{\fdu_thm_new_no_number:nnn,
%   \fdu_thm_new_no_number:Vxx}
% 不带编号的定理环境。
% \begin{arguments}
%   \item 样式
%   \item 定理环境名称
%   \item 定理头文字
% \end{arguments}
%    \begin{macrocode}
\cs_new:Npn \fdu_thm_new_no_number:nnn #1#2#3
  {
    \@@_thm_ntheorem_style:n {#1}
    \@@_thm_ntheorem_new:w   {#2} {#3}
  }
\cs_generate_variant:Nn \fdu_thm_new_no_number:nnn { Vxx }
%    \end{macrocode}
% \end{macro}
%
% \begin{macro}{\fdu_thm_set_qed:n,
%   \fdu_thm_set_header_font:n,
%   \fdu_thm_set_body_font:n,
%   \fdu_thm_set_qed:V,
%   \fdu_thm_set_header_font:V,
%   \fdu_thm_set_body_font:V}
% 封装 \pkg{ntheorem} 宏包提供的若干命令,分别用以设置证毕符号、
% 定理头字体和定理正文字体。
%    \begin{macrocode}
\cs_new:Npn \fdu_thm_set_qed:n         #1 { \theoremsymbol     {#1} }
\cs_new:Npn \fdu_thm_set_header_font:n #1 { \theoremheaderfont {#1} }
\cs_new:Npn \fdu_thm_set_body_font:n   #1 { \theorembodyfont   {#1} }
\cs_generate_variant:Nn \fdu_thm_set_qed:n         { V }
\cs_generate_variant:Nn \fdu_thm_set_header_font:n { V }
\cs_generate_variant:Nn \fdu_thm_set_body_font:n   { V }
%    \end{macrocode}
% \end{macro}
%
% \end{implementation}
%
