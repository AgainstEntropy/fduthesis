% \iffalse meta-comment
% !TeX program  = XeLaTeX
% !TeX encoding = UTF-8
%
%<*internal>
\iffalse
%</internal>
%
%<*internal>
\fi
\begingroup
  \def\NameOfLaTeXe{LaTeX2e}
\expandafter\endgroup\ifx\NameOfLaTeXe\fmtname\else
\csname fi\endcsname
%</internal>
%
%<*internal>
\fi
%</internal>
%
%<*driver>
\PassOptionsToPackage{showframe}{geometry}
\documentclass{fdudoc}
\hypersetup
  {
    pdftitle  = {fduthesis: 复旦大学论文模板},
    pdfauthor = {曾祥东}
  }
\renewcommand*\expstar{\hyperlink{expstar}{$\mfrake$}}
\renewcommand*\rexpstar{\hyperlink{rexpstar}{$\mfrakc$}}
\begin{document}
  % \DisableImplementation
  \EnableImplementation
  \DocInput{dtxtest.dtx}
  \IndexLayout
  \PrintChanges
  \PrintIndex
\end{document}
%</driver>
% \fi
%
% \changes{v0.1}{2017/02/15}{开始编写模板。}
% \changes{v0.2}{2017/02/19}{使用 Git 进行版本控制,并发布至 GitHub。}
%
% \CheckSum{0}
%
% \CharacterTable
%  {Upper-case    \A\B\C\D\E\F\G\H\I\J\K\L\M\N\O\P\Q\R\S\T\U\V\W\X\Y\Z
%   Lower-case    \a\b\c\d\e\f\g\h\i\j\k\l\m\n\o\p\q\r\s\t\u\v\w\x\y\z
%   Digits        \0\1\2\3\4\5\6\7\8\9
%   Exclamation   \!     Double quote  \"     Hash (number) \#
%   Dollar        \$     Percent       \%     Ampersand     \&
%   Acute accent  \'     Left paren    \(     Right paren   \)
%   Asterisk      \*     Plus          \+     Comma         \,
%   Minus         \-     Point         \.     Solidus       \/
%   Colon         \:     Semicolon     \;     Less than     \<
%   Equals        \=     Greater than  \>     Question mark \?
%   Commercial at \@     Left bracket  \[     Backslash     \\
%   Right bracket \]     Circumflex    \^     Underscore    \_
%   Grave accent  \`     Left brace    \{     Vertical bar  \|
%   Right brace   \}     Tilde         \~}
%
% \title{\textcolor{MaterialIndigo800}{^^A
%   \textbf{fduthesis: 复旦大学论文模板}}}
% \author{曾祥东}
% \date{\today \quad v0.6^^A
%   \thanks{\url{https://github.com/stone-zeng/fduthesis}.}}
%
%^^A 禁止使用 " 符号作为抄录文本缩略符
% \DeleteShortVerb\"
%
%^^A 封面与目录的页边距
% \newgeometry{
%   left   = 1.25 in,
%   right  = 1.25 in,
%   top    = 1.25 in,
%   bottom = 1.00 in
% }
%
% \maketitle
% \tableofcontents
%
% \EnableDocumentation
%
% \begin{documentation}
%
%^^A 用户手册的页边距
% \newgeometry{
%   left   = 1.75 in,
%   right  = 1.00 in,
%   top    = 1.25 in,
%   bottom = 1.00 in
% }
%
% \section{介绍}
%
% 本模板将借鉴前辈经验,重新设计,并使用 \LaTeX3
% \scite{interfaces3} 编写,以适应 \TeX{} 技术发展潮流;
% 同时还将构建一套简洁的接口,方便用户使用。
%
% 本模板中的选项、命令或环境可以分为以下三类:
% \scite{knuth1986texbook,interfaces3,richard2016fudanthesis}
% \begin{itemize}
%   \item 名字后面带有 \rexptarget\rexpstar{} 的,表示只能在^^A
%     \emph{中文模板}中使用;
%   \item 名字后面带有 \exptarget\expstar{} 的,表示只能在^^A
%     \emph{英文模板}中使用;
%   \item 名字后面不带有特殊符号的,表示既可以在中文模板中使用,
%     也可以在英文模板中使用。
% \end{itemize}
%
% \subsection{模板组成}
%
% 如需生成用户手册 \file{fduthesis.pdf},可在命令行中执行
% \begin{shellexample}[morekeywords={xelatex,makeindex},emph={-o,-s,-t}]
%   xelatex fduthesis.dtx
%   makeindex -s gind.ist -o fduthesis.ind fduthesis.idx
%   makeindex -s gglo.ist -o fduthesis.gls -t fduthesis.glg fduthesis.glo
%   xelatex fduthesis.dtx
%   xelatex fduthesis.dtx
% \end{shellexample}
% 也可使用 \pkg{latexmk}:
% \begin{shellexample}[morekeywords={latexmk}]
%   latexmk fduthesis.dtx
% \end{shellexample}
% 本模板已经为编译用户手册提供了 \pkg{latexmk} 配置文件
% \file{latexmkrc}。
%
% \section{使用说明}
%
% \subsection{基本用法}
%
% 以下是一份简单的 \TeX{} 文档,它演示了 \cls{fduthesis}
% 的最基本用法:
% \begin{latexexample}[deletetexcs={\documentclass},%
%     moretexcs={\chapter},morekeywords={\documentclass},%
%     emph={[2]document}]
%   % thesis.tex
%   \documentclass{fduthesis}
%   \begin{document}
%     \chapter{您好}
%     \section{Welcome to fduthesis!}
%     你好,\LaTeX{}!
%   \end{document}
% \end{latexexample}
%
% \subsection{模板选项}
%
% 所谓“模板选项”,指需要在引入文档类的时候指定的选项:
% \begin{latexexample}[deletetexcs={\documentclass},%
%     morekeywords={\documentclass}]
%   \documentclass(*\oarg{模板选项}*){fduthesis}
%   \documentclass(*\oarg{模板选项}*){fduthesis-en}
% \end{latexexample}
%
% 有些模板选项为布尔型,它们只能在 \opt{true} 和 \opt{false}
% 中取值。对于这些选项,\kvopt{\meta{选项}}{true} 中的“|= true|”
% 可以省略。
%
% \begin{function}[added=2018-02-01]{type}
%   \begin{fdusyntax}[emph={[1]type}]
%     type = (*<doctor|master|(bachelor)>*)
%   \end{fdusyntax}
%   选择论文类型。三种选项分别代表博士学位论文、硕士学位论文和本科
%   毕业论文。
% \end{function}
%
% \begin{function}{oneside,twoside}
%   指明论文的单双面模式,默认为 \opt{twoside}。该选项会影响每章
%   的开始位置,还会影响页眉样式。
% \end{function}
%
% 在双面模式(\opt{twoside})下,按照通常的排版惯例,每章应只从
% 奇数页(在右)开始;而在单页模式(\opt{oneside})下,则可以从
% 任意页面开始。本模板中,目录、摘要、符号表等均视作章,也按相同
% 方式排版。
%
% \subsection{参数设置}
%
% \begin{function}{\fdusetup}
%   \begin{fdusyntax}[morekeywords={\fdusetup}]
%     \fdusetup(*\marg{键值列表}*)
%   \end{fdusyntax}
%   本模板提供了一系列选项,可由您自行配置。载入文档类之后,以下
%   所有选项均可通过统一的命令 \cs{fdusetup} 来设置。
% \end{function}
%
% \cs{fdusetup} 的参数是一组由(英文)逗号隔开的选项列表,列表中的
% 选项通常是 \kvopt{\meta{key}}{\meta{value}} 的形式。部分选项的
% \meta{value} 可以省略。对于同一项,后面的设置将会覆盖前面的设置。
% 在下文的说明中,将用\textbf{粗体}表示默认值。
%
% 另有一些选项包含子选项,如 \opt{style} 和 \opt{info} 等。它们可以
% 按如下两种等价方式来设定:
% \begin{latexexample}[morekeywords={\fdusetup},%
%     emph={[1]style,cjkfont,fontsize,info,title,title*,author,author*,department}]
%   \fdusetup{
%     style = {cjkfont = adobe, fontsize = -4},
%     info  = {
%       title      = {关于光产生和转变的一个启发性观点},
%       title*     = {On a Heuristic Viewpoint Concerning the Production
%         and Transformation of Light},
%       author     = {阿尔伯特·爱因斯坦},
%       author*    = {Albert Einstein},
%       department = {物理学系}
%     }
%   }
% \end{latexexample}
% 或者
% \begin{latexexample}[morekeywords={\fdusetup},%
%     emph={[1]style,cjkfont,fontsize,info,title,title*,author,author*,department}]
%   \fdusetup{
%     style/cjkfont   = adobe,
%     style/fontsize  = -4,
%     info/title      = {关于光产生和转变的一个启发性观点},
%     info/title*     = {On a Heuristic Viewpoint Concerning the Production
%       and Transformation of Light},
%     info/author     = {阿尔伯特·爱因斯坦},
%     info/author*    = {Albert Einstein},
%     info/department = {物理学系}
%   }
% \end{latexexample}
%
% 注意 “|/|” 的前后均不可以出现空白字符。
%
% \subsection{正文编写}
%
% \begin{quotation*}[喬孟符][宋]
%   作樂府亦有法,曰\CJKunderdot{鳳頭豬肚豹尾}六字是也。
%   大概起要美麗,中要浩蕩,結要響亮。尤貴在首尾貫穿,意思清新。
%   茍能若是,斯可以言樂府矣。
% \end{quotation*}
%
% \subsubsection{凤头}
%
% \begin{function}{\frontmatter}
%   声明前置部分开始。
% \end{function}
%
% 在本模板中,前置部分包含目录、中英文摘要以及符号表等。
% 前置部分的页码采用小写罗马字母,并且与正文分开计数。
%
% \begin{function}{\tableofcontents}
%   生成目录。为了生成完整、正确的目录,您至少需要编译\emph{两次}。
% \end{function}
%
%^^A TODO: \DescribeEnv{abstract}
%^^A TODO: \DescribeEnv{abstract*}
% \begin{function}{abstract}
%   \begin{fdusyntax}[emph={[2]abstract}]
%     % 中文论文模板 (fduthesis)      % 英文论文模板 (fduthesis-en)
%     \begin{abstract}                \begin{abstract}
%       (*\meta{中文摘要} \hspace{3.52cm} \meta{英文摘要}*)
%     \end{abstract}                  \end{abstract}
%   \end{fdusyntax}
% \end{function}
% \begin{function}[rEXP]{abstract*}
%   \begin{fdusyntax}[emph={[2]abstract*}]
%     % 中文论文模板 (fduthesis)
%     \begin{abstract*}
%       (*\meta{英文摘要}*)
%     \end{abstract*}
%   \end{fdusyntax}
%   摘要。中文模板中,不带星号和带星号的版本分别用来输入中文摘要
%   和英文摘要;英文模板中没有带星号的版本,您只需输入英文摘要。
% \end{function}
%
% 摘要的最后,会显示关键字列表以及中图分类号(CLC)。
% 这两项可通过 \cs{fdusetup} 录入。
%
% \begin{thebibliography}{9}
%
% \providecommand{\urlprefix}{\newline\hspace*{\fill}}
% \let\OldUrl=\url
% \renewcommand\url[1]{{\small\OldUrl{#1}}}
% \newcommand\sourceurl[1]{{\small{\kaishu 源代码:}
%   \href{https://www.ctan.org/tex-archive/#1}{\ttfamily CTAN://#1}}}
% \newcommand\CTANurl[1]{{\small\href{https://www.ctan.org/tex-archive/#1}^^A
%   {\ttfamily CTAN://#1}}}
%
% \bibitem{knuth1986texbook}
% \textsc{Knuth D E}.
% \newblock \textit{The \TeX book: Computers \& Typesetting, volumn A} [M].
% \newblock Boston: Addison--Wesley Publishing Company, 1986
%   \urlprefix {\small {\kaishu 源代码
%     \footnote{此代码只可作为学习之用。未经 Knuth 本人同意,您不应当编译此文档。}:}
%     \CTANurl{systems/knuth/dist/tex/texbook.tex}}
%
% \bibitem{interfaces3}
% \textsc{The \LaTeX3 Project}.
% \newblock \textit{The \LaTeX3 Interfaces} [EB/OL].
% \newblock (2017-11-14) \urlprefix
%   \CTANurl{macros/latex/contrib/l3kernel/interface3.pdf}
%
% \bibitem{richard2016fudanthesis}
% \textsc{richard}.
% \newblock \textit{复旦大学硕士学位论文模板} [EB/OL].
% \newblock (2016-01-31) \urlprefix
%   \url{https://github.com/richarddzh/fudan-thesis}
%
% \end{thebibliography}
%
% \clearpage
%
% \end{documentation}
%
% \begin{implementation}
%
% \newgeometry{
%   left      = 2.25 in,
%   right     = 1.00 in,
%   top       = 1.25 in,
%   bottom    = 1.00 in,
%   marginpar = 2.25 in
% }
%
% \section{实现细节}
%
% 本模板使用 \LaTeX3 语法编写,依赖 \pkg{expl3} 环境,
% 并需调用 \pkg{l3packages} 中的相关宏包。
%
% \subsection{准备}
%
% \subsubsection{内部变量声明}
%
%    \begin{macrocode}
%<@@=fdu>
%<*class|class-en>
%    \end{macrocode}
%
% \begin{variable}{\l_@@_tmpa_box,
%   \l_@@_tmpa_dim,\l_@@_tmpb_dim,
%   \l_@@_tmpa_tl,\l_@@_tmpb_tl,
%   \l_@@_tmpa_clist,\l_@@_tmpb_clist}
% 临时变量。
%    \begin{macrocode}
\box_new:N   \l_@@_tmpa_box
\dim_new:N   \l_@@_tmpa_dim
\dim_new:N   \l_@@_tmpb_dim
\tl_new:N    \l_@@_tmpa_tl
\tl_new:N    \l_@@_tmpb_tl
\clist_new:N \l_@@_tmpa_clist
\clist_new:N \l_@@_tmpb_clist
%    \end{macrocode}
% \end{variable}
%
% \begin{variable}{\g_@@_thesis_type_int}
% 论文类型。取值 1、2、3 分别对应博士、硕士、本科(学士),这与学号
% 第三位是一致的。
%    \begin{macrocode}
\int_new:N \g_@@_thesis_type_int
%    \end{macrocode}
% \end{variable}
%
% \subsubsection{内部函数}
%
% \begin{macro}{\cs_generate_variant:cn,
%   \file_input:V,
%   \int_to_arabic:v,
%   \keys_define:nx}
% \begin{macro}[TF]{\tl_if_eq:Vn}
% \LaTeX3{} 函数变体。
%    \begin{macrocode}
\cs_generate_variant:Nn \cs_generate_variant:Nn { cn }
\cs_generate_variant:Nn \file_input:n           { V  }
\cs_generate_variant:Nn \int_to_arabic:n        { v  }
\cs_generate_variant:Nn \keys_define:nn         { nx }
\prg_generate_conditional_variant:Nnn \tl_if_eq:nn { Vn } { T, TF }
%    \end{macrocode}
% \end{macro}
% \end{macro}
%
% \begin{macro}{\@@_quad:,\@@_qquad:}
% 等价于 \LaTeXe{} 中的 \tn{quad} 和 \tn{qquad}。
%    \begin{macrocode}
\cs_new:Npn \@@_quad:  { \skip_horizontal:n { 1 em } }
\cs_new:Npn \@@_qquad: { \skip_horizontal:n { 2 em } }
%    \end{macrocode}
% \end{macro}
%
% \subsection{选项处理}
%
% 定义 |fdu/option| 键值类。
%    \begin{macrocode}
\keys_define:nn { fdu / option }
  {
%    \end{macrocode}
%
% \changes{v0.7}{2018/02/01}{新增 \opt{type} 选项。}
%
% \begin{macro}{type}
% 设置论文类型。设为模板选项主要是为了以后的兼容性。论文类型可能会
% 影响很多设置,只是暂时还不考虑。默认为本科毕业论文。
%    \begin{macrocode}
    type .choice:,
    type .value_required:n = true,
    type .choices:nn =
      { doctor, master, bachelor }
      { \int_set_eq:NN \g_@@_thesis_type_int \l_keys_choice_int },
    type .initial:n = bachelor,
%    \end{macrocode}
% \end{macro}
%
% \changes{v0.7}{2018/01/31}{新增 \opt{config} 选项。}
%
% \begin{macro}{config}
% 配置文件名。
%    \begin{macrocode}
    config .tl_set:N = \g_@@_config_tl,
%    \end{macrocode}
% \end{macro}
%
% 处理未知选项。
%    \begin{macrocode}
    unknown .code:n = { \@@_error:n { unknown-option } }
  }
\@@_msg_new:nn { unknown-option }
  { Class~ option~ "\l_keys_key_tl"~ is~ unknown. }
%    \end{macrocode}
%
% 将文档类选项传给 |fdu/option|。
%    \begin{macrocode}
\ProcessKeysOptions { fdu / option }
%</class|class-en>
%    \end{macrocode}
%
% \subsection{摘要}
%
% \begin{environment}{abstract}
% \begin{environment}{abstract*}
% \changes{v0.7}{2018/03/05}{整理代码。}
% 摘要环境。在中文模板定义了中英文双语摘要,但在英文模板中则没有
% 定义中文摘要。
%    \begin{macrocode}
\NewDocumentEnvironment { abstract  } { }
%<class>  { \@@_abstract_begin:    } { \@@_abstract_end:      }
%<class-en>  { \@@_abstract_en_begin: } { \@@_abstract_en_end:   }
%<*class>
\NewDocumentEnvironment { abstract* } { }
  { \@@_abstract_en_begin: } { \@@_abstract_en_end:   }
%</class>
%    \end{macrocode}
% \end{environment}
% \end{environment}
%
% \begin{environment}{function}
% \begin{macro}{\@@_fix_previous_depth:}
% 调整 \env{function} 环境前后间距。
%    \begin{macrocode}
\BeforeBeginEnvironment { function }
  { \par \nointerlineskip }
\AtEndEnvironment { function }
  {
    \par
    \cs_gset:Nx \@@_fix_previous_depth:
      { \prevdepth = \the \prevdepth \space }
  }
\AfterEndEnvironment { function }
  { \@@_fix_previous_depth: }
%    \end{macrocode}
% \end{macro}
% \end{environment}
%
% \clearpage
%
% \end{implementation}
%
