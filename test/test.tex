\input fduthesis-test-toolkit

\documentclass[twoside]{fduthesis}
\usepackage{kantlipsum}
\usepackage{zhlipsum}

\fdusetup{
  style = {
    font = times,
    cjk-font = fandol,
    font-size = -4,
    fullwidth-stop = false,
    logo-name   = {../testfiles/support/fudan-name.pdf},
    logo-emblem = {../testfiles/support/fudan-emblem.pdf},
    hyperlink = color,
    hyperlink-color = default,
%
%
    bib-backend = bibtex,
    bib-style = numerical,
    bib-resource = {test.bib}
  },
  info = {
    title = {{自河南经乱关内阻饥兄弟离散各在一处}, {因望月有感聊书所怀}},
    % title = {自河南经乱关内阻饥兄弟离散各在一处因望月有感聊书所怀},
%    date = {2017年2月10日},
    author = {某某某},
    supervisor = {Alpha Beta},
    supervisor-title = {研究员},
    major = {物理学},
    department = {凝聚态物理系},
    affiliation = {复旦大学},
    student-id = {14307110000},
    keywords = {\LaTeX, 排版, 字体排印, 学位论文, 数学, 物理, 计算机},
    keywords* = {\LaTeX, typesetting, typography, dissertation, mathematics, physics, computer science},
    clc = {O414.1/65}
  }
}

\def\WORLDb{你好,世界。}

\newcommand{\fonttesttext}{你好,世界\symbol{12290}Hello, world! fi fl ff ffi ffl fff}
\newcommand\fonttest{%
  正常:\qquad\fonttesttext \par
  粗体:\qquad\textbf{\fonttesttext} \par
  倾斜:\qquad\textit{\fonttesttext} \par
  粗斜:\qquad\textbf{\textit{\fonttesttext}} \par
  小型大写:\qquad\textsc{\fonttesttext}
}

\def\BibTeX{B\textsc{ib}\TeX}

\begin{document}

\raggedbottom

\frontmatter

\tableofcontents

\begin{abstract}
\zhlipsum[1-10]
\end{abstract}

\begin{abstract*}
\kant[1-10]
\end{abstract*}

\begin{notation}[lp{20em}]
$\sin$      &  正弦 \\
HPC         &  高性能计算 (High Performance Computing) \\
cluster     &  集群 \\
Itanium     &  安腾 \\
SMP         &  对称多处理 \\
API         &  应用程序编程接口 \\
PI          &  聚酰亚胺 \\
MPI         &  聚酰亚胺模型化合物,N-苯基邻苯酰亚胺 \\
PBI         &  聚苯并咪唑 \\
MPBI        &  聚苯并咪唑模型化合物,N-苯基苯并咪唑 \\
PY          &  聚吡咙 \\
PMDA-BDA    &  均苯四酸二酐与联苯四胺合成的聚吡咙薄膜 \\
$\Delta G$  &  活化自由能 (Activation Free Energy) \\
$\chi$      &  传输系数 (Transmission Coefficient) \\
$E$         &  能量 \\
$m$         &  质量 \\
$c$         &  光速 \\
$P$         &  概率 \\
$T$         &  时间 \\
$v$         &  速度
\end{notation}

\mainmatter

\chapter{文本,字体,脚注 \quad Text, font and footnote}

\section{文字与段落 Text and paragraph}

\subsection{中文文本 Chinese}
\zhlipsum

\subsection{英文文本 English}
\kant

\clearpage

\section{字体 Font}

\subsection{普通字体 Roman}
\fonttest

\subsection{无衬线字体 Sans-serif}
\textsf{\fonttest}

\subsection{打字机字体 Typewriter}
\texttt{\fonttest}

\subsection{句号}
如果\symbol{"002E}会突然:\textsc{Full Stop} \par
如果\symbol{"3002}会突然:\textsc{Ideographic Full Stop} \par
如果\symbol{"FF0E}会突然:\textsc{Fullwidth Full Stop} \par
如果\symbol{"FF61}会突然:\textsc{Halfwidth Ideographic Full Stop} \par

\subsection{字号}
字号{\zihao{4}四号}
字号{\zihao{-4}小四}
字号{\zihao{5}五号}

\input{test-footnote}

\input{test-theorem}

\chapter{图表 vs 浮动体}

\section{title}
Myriad,英语单词,意为「无数的」。同时,「Myriad」也是一款字体的名字。
由罗伯特·斯林巴赫(Robert Slimbach,1956年-)和卡罗·图温布利
(Carol Twombly,1959年-)在1990年到1992年期间以 Frutiger 字体为蓝本
为 Adobe 公司设计。 Myriad 是早期数码字体时代的先驱,伴随着技术的成长
一路走来。

\begin{figure}[h]
  \centering
  \includegraphics[width=3cm]{../testfiles/support/fudan-emblem.pdf}
  \includegraphics[width=4cm]{../testfiles/support/fudan-emblem-new.pdf}
  \caption{Multiple Master 是 Type 1字体格式的扩展部分。Type 1 是利用
    PostScript 语言描述字形信息的字体系统。Type 1字体是第一款矢量字体
    (outline font),通过二维坐标系中的关键点和三次贝塞尔曲线描述字体
    的边缘,在屏幕显示和输出时,在光栅图像处理器内,根据字号大小计算
    出字体边缘(栅格化)。}
\end{figure}

如今,它更多地和我们相见在显示屏幕上。当然,还有那著名的标榜设计的
电子品牌。1992 年,耗时两年开发的 Myriad 终于发布了历史上第一个版本:
Myriad MM。

\section{title}
这款温和且具有良好可读性的人文主义无衬线字体,集诸多当时最新的数字
字体技术于一身。 后缀 MM,意为 Multiple Master,没有找到对应的中文
译名,我们权且称之为「多母板技术」。Myriad 是最早采用 Multiple Master
技术的无衬线字体之一。这项技术的原理是在坐标轴(Axis)的区间两端设计
极限母板,中间的变量则采取线性或非线性变化,对于字体来说,字型的宽度、
粗细甚至有无衬线,都可以在坐标轴上设置。此外,MM 技术还提供了在小字号
下屏幕显示的视觉修正(Optical Adjustment),也就是说,同一款字体,在
小字号时,其字间距和笔画粗细,会被适当地放大。而衬线字体,随着字号的
变小,衬线会相对变粗。视觉修正可以提高小字号字体的识别性,对于远低于
印刷分辨率的电脑屏幕来说,也具有重要意义。

\begin{table}[h]
  \centering
  \caption{一个 normal 表格}
  \begin{tabular}{ccc}
    \hline
    \bfseries 功能 & \bfseries 环境 & \bfseries code \\
    \hline
    表格 & tabular & \ttfamily \backslash begin\{tabular\} ... \backslash end\{tabular\} \\
    插图 & figure  & \ttfamily \backslash begin\{figure\}  ... \backslash end\{figure\}  \\
    居中 & center  & \ttfamily \backslash begin\{center\}  ... \backslash end\{center\}  \\
    \hline
  \end{tabular}
\end{table}

在 Multiple Master 的时代,字号是从6pt到72pt之间非线性设置的。这一传统
保留到了今天 Truetype 和 Opentype 的 Single Master 时代。Adobe 软件的
字体下拉菜单,仍然只显示6到72pt 的字号。


\chapter{文本}

\section{文字与段落}

\textbf{本段使用 \texttt{\string\cite}}
Myriad,英语单词,意为「无数的」\cite{sunstein,gjhjbhjkjbzs,hblzsthjkjyxgs}。
同时,「Myriad」也是一款字体的名字。
由罗伯特·斯林巴赫(Robert Slimbach,1956年--)和卡罗·图温布利
(Carol Twombly,1959年-)\cite{wyf,sunstein,zgtsgxh,lzp1}
在1990年到1992年期间以 Frutiger 字体为蓝本为 Adobe 公司设计\cite{cdy}。
Myriad 是早期数码字体时代的先驱,\cite{wfz,wfz1}
伴随着技术的成长一路走来 \cite{hlswedl,zgdylsdag}。

\textbf{本段使用 \texttt{\string\citep}}
如今,它更多地和我们相见在显示屏幕上 \citep{wfz2}。当然,还有那著名的标榜设计的
电子品牌 \citep{cgw,mks}。1992 年,耗时两年开发的 Myriad 终于发布了历史上第一个版本:
Myriad MM \citep{wyf,hblzsthjkjyxgs,sunstein,zgtsgxh,aaas}。

\section{title}

\textbf{本段使用 \texttt{\string\citet}}
这款温和且具有良好可读性的人文主义无衬线字体\citet{yjb},集诸多当时最新的数字
字体技术于一身。 后缀 MM,意为 Multiple Master,没有找到对应的中文
译名\citet{lbm,calkin},我们权且称之为「多母板技术」。Myriad 是最早采用 Multiple Master
技术的无衬线字体之一。这项技术的原理是在坐标轴(Axis)的区间两端设计
极限母板,中间的变量则采取线性或非线性变化,对于字体来说,字型的宽度、
粗细甚至有无衬线\citet{xadzkjdx,yufin,cgw},都可以在坐标轴上设置。此外,MM 技术还提供了在小字号
下屏幕显示的视觉修正(Optical Adjustment),也就是说,同一款字体,在
小字号时,其字间距和笔画粗细,会被适当地放大。而衬线字体,随着字号的
变小,衬线会相对变粗。视觉修正可以提高小字号字体的识别性,对于远低于
印刷分辨率的电脑屏幕来说,也具有重要意义。

在 Multiple Master 的时代,字号是从6pt到72pt之间非线性设置的。这一传统
保留到了今天 Truetype 和 Opentype 的 Single Master 时代。Adobe 软件的
字体下拉菜单,仍然只显示6到72pt 的字号。

\printbibliography

\end{document}
